% Copyright 2019 by Till Tantau
%
% This file may be distributed and/or modified
%
% 1. under the LaTeX Project Public License and/or
% 2. under the GNU Free Documentation License.
%
% See the file doc/generic/pgf/licenses/LICENSE for more details.


\section{Introducción}

Bienvenido a la documentación de \tikzname\ y el sistema subyacente \pgfname. Lo
que empezó como un pequeño estilo de \LaTeX\ para crear los gráficos
de mi (Till Tantau) tesis de doctorado directamente con pdf\LaTeX\ se ha
convertido en un lenguaje para la creación de gráficos completo, con un manual
de más de míl páginas. La gran cantidad de opciones que ofrece \tikzname\
suele ser abrumadora para los principiantes; pero, afortunadamente, esta
documentación viene con varios tutoriales detallados que te enseñarán casi
todo lo que deberías saber para usar \tikzname\, sin que tengas que leer el
resto. 

Me gustaría inicial con la pregunta de ``¿Qué es \tikzname?'' Básicamente, es
solo un conjunto de comandos de \TeX\ que dibujan figuras. Por ejemplo, el
código |\tikz \draw (0pt,0pt) -- (20pt,6pt);| dibuja la línea \tikz \draw
(0pt,0pt) -- (20pt,6pt); mientras que el código |\tikz \fill[orange] (1ex,1ex) circle (1ex);|
dibuja \tikz \fill[orange] (1ex,1ex) circle (1ex);. En cierto sentido, cuando
usas \tikzname\ lo que estás haciendo es que ``programas'' tu gráfico, de la
misma manera en la que ``programas'' tu documento cuando usas \TeX. Esto también
explica el nombre: \tikzname es un acrónimo recursivo al estilo de
``\textsc{gnu}'s Not Unix'' y que significa ``\tikzname\ ist \emph{kein}
Zeichenprogramm'', que se traduce como ``\tikzname\ no es un programa de
dibujo'', advirtiendo al lector sobre lo que debe esperar. Con \tikzname\ 
obtendrás todas las ventajas del ``Enfoque de composición tipográfica de \TeX''
para tus gráficos: rápida creación de gráficos simples, posicionamiento preciso,
el uso de macros, una composición tipográfica usualmente superior. Sin embargo
también heredando todas sus desventajas: una curva de aprendizaje pronunciada,
no \textsc{wysiwyg}\footnote{Nota de traducción:\textsc{wysiwyg} se refiere a la
frase ``what you see is what you get'' que se traduce como ``lo que ves es lo
que obtienes'' y se refiere a los programas de edición en los cuales el
resultado que ves en pantalla es el mismo que el producto final, como MS Word
o Adobe Photoshop.}, pequeños cambios requieren un largo tiempo de compilación, 
y que el código no realmente ``muestra'' como las cosas se van a ver. 

Ahora que sabemos qué es \tikzname\ ¿que hay acerca de ``\pgfname''? Como se
mencionó anteriormente,\tikzname\ empezó como un proyecto para implementar
macros de gráficos de \TeX que pudieran ser usados por pdf\LaTeX y también con
el clásico \LaTeX (basado en PostScript). En otras palabras, lo que quería era
implementar un ``formato de gráficos portable'' (portable graphics format) para
\TeX\ -- de ahí el nombre de \pgfname. Esos primeros macros están aún ahí y
forman la ``capa básica'' del sistema descrito en este manual, pero actualmente
la mayoría de la interacción que un autor tendrá será con \tikzname\ -- el cual
se ha convertido en un lenguaje completamente propio.


\subsection{Las Capas Debajo de \tikzname}

Resulta que en realidad hay \emph{dos} capas debajo de \tikzname:
%
\begin{description}
    \item[Capa del Sistema:] Esta capa proporciona una abstracción completa de
    lo que sucede ``en el driver''. El driver es un programa como |dvips| o
    |dvipdfm| que toma un archivo |.dvi| como entrada y genera un archivo |.ps|
    o |.pdf|. (El programa |pdftex| también cuenta como un driver aunque no
    tome un archivo |.dvi| como entrada. No es de importancia.) Cada driver
    tiene su propio sintaxis para la creación de figuras, lo que causa dolores
    de cabeza a cualquiera que quiera crear gráficos de forma portable. La capa
    del sistema de \pgfname\ es la que ``abstrae'' estas diferencias. Por
    ejemplo, el comando del sistema |\pgfsys@lineto{10pt}{10pt}| extiende el
    path activo a la coordenada $(10\mathrm{pt},10\mathrm{pt})$ del
    |{pgfpicture}| en ejecución. Dependiendo de si |divps|, |dvipdfm| o |pdftex|
    es usado para procesar el documento, el comando del sistema será convertido
    a distintos comandos |\special|. La capa del sistema es lo más minimalista
    posible, ya que cada comando hace que sea más laborioso portar \pgfname\ a
    un nuevo driver.

    Como usuario, no utilizarás la capa del sistema directamente.

    \item[Capa Básica:] La capa básica proporciona un conjunto de comandos
    básicos que permiten producir gráficos complejos de manera mucho más
    sencilla que usando directamente la capa del sistema. Por ejemplo, la capa
    del sistema no proporciona comandos para crear círculos, dado que los
    círculos pueden ser creados mediante las, más básicas, curvas de Bézier
    (bueno, casi). Sin embargo, como usuario querrás tener un comando simple 
    para crear círculos (al menos yo sí) en vez de tener que escribir media
    página de coordenadas de control para la curva de Bézier. De esta manera, la
    capa básica proporciona un comando |\pgfpathcircle| que genera por ti las
    coordenadas necesarias para la curva. 

    La capa básica consiste de un \emph{núcleo}, el cual consiste de varios
    paquetes interdependientes que solo pueden ser cargados \emph{en bloc}, y de
    \emph{módulos} adicionales que extienden el núcleo mediante comandos de
    propósitos más específicos, como los del manejo de los nodos o una interfaz
    para graficar. Por ejemplo, el paquete \textsc{beamer} solo usa el núcleo y
    no, por ejemplo, el módulo |shapes|.

\end{description}

En teoría, \tikzname\ por si mismo es solo uno de los muchos ``frontends''
posibles, los cuales son un conjunto de comandos o una sintaxis especial que
hacen más sencillo usar la capa básica. Un problema de usar la capa básica
directamente es que el código escrito por esta capa suele ser demasiado
``verboso''. Por ejemplo, para dibujar un simple triángulo, podrías necesitar
hasta cinco comandos cuando usad la capa básica: Uno para iniciar el path en el
primer vértice del triángulo, uno para extender el path hasta el segundo
vértice, uno para ir al tercero, uno para cerrar el path, y uno para realmente
dibujar el triángulo (en lugar de rellenarlo). Con el frontend \tikzname\  todo
esto se reduce a un único y sencillo comando similar a \textsc{metafont}:
%
\begin{verbatim}
\draw (0,0) -- (1,0) -- (1,1) -- cycle;
\end{verbatim}

En la práctica, \tikzname\ es el único frontend ``serio'' para \pgfname. Este te
permite acceder a todas las características de \pgfname, pero con un diseño
fácil de usar. La sintaxis es una mezcla de \textsc{metafont}, \textsc{pstricks}
y algunas de mis propias ideas. Existe otros frontends además de \tikzname, pero
estos están pensados más como ``estudios de tecnología'' y menos como
alternativas serias de \tikzname. En particular, el frontend |pgfpict2e|
reimplementa el entorno estándar de \LaTeX\ |{picture}| y los comandos como
|\line| o |\vector| usando la capa básica de \pgfname. Esta capa no es realmente
necesaria, ya que el paquete |pict2e.sty| hace un trabajo igual de bueno en
reimplementar el entorno |{picture}|. Más bien, la idea detrás de este paquete
es tener una demostración simple de como un frontend puede ser implementado.

Dado que la mayoría de los usuarios solo usarán \tikzname\ y que casi nadie
usará la capa del sistema directamente, este manual es principalmente acerca de
\tikzname\ en las primeras partes; la capa básica y la capa del sistema serán
explicados al final.

\subsection{Comparación con Otros Paquetes Gráficos}

\tikzname\ no es el único paquete gráfico para \TeX. A continuación, trataré de
dar una razonablemente comparación justa de \tikzname\ y otros paquetes gráficos
%
\begin{enumerate}
    \item El entorno |{picture}| estándar de \LaTeX\ te permite crear gráficos
    simples, pero poco más. Esto ciertamente no es debido a una falta de
    conocimiento o imaginación por parte de los diseñadores de \LaTeX. Más bien,
    este es el precio a pagar por la portabilidad del entorno |{picture}|: este
    funciona con todos los drivers de backend.
    
    \item El paquete |{pstricks}| ciertamente es lo suficientemente potente para
    crear cualquier tipo de gráfico concebible, pero no es realmente portable.
    Más importante, este no funciona con |pdftex| ni con cualquier otro driver
    que produzca algo distinto a código PostScript.

    Comparado con \tikzname, |pstricks| tiene una base de apoyo similar. Hay
    bastantes buenos paquetes extra para situaciones especiales que han sido
    contribuidos por los usuarios durante la última década. La sintaxis de
    \tikzname\ es más consistente que la de |pstricks|, dado que \tikzname ha
    sido desarrollado ``de forma más centralizada'' y también ``teniendo en
    cuente las deficiencias de |pstricks|''.

    \item El paquete |xypic| es un paquete antiguo para crear gráficos. Sin
    embargo, es más difícil de usar y de aprender dado que la sintaxis y la
    documentación son algo crípticas.

    \item El paquete |dratex| es un pequeño paquete para crear gráficos.
    Comparado con los otros paquetes, incluyendo \tikzname, es muy pequeño, lo
    que podría o no ser una ventaja.

    \item El programa |metapost| es una poderosa alternativa a \tikzname. Solía
    ser un programa externo, lo cual traía consigo un montón de problemas, pero
    con Lua\TeX\ está ahora integrado. Un obstáculo con |metapost| es la
    inclusión de etiquetas. Esto es \emph{mucho más} sencillo de lograr usando
    \pgfname.

    \item El programa |xfig| es una alternativa importante a \tikzname\ para los
    usuarios que no desean ``programar'' sus gráficos, dado que esto es
    necesario con \tikzname\ y los otros paquetes anteriores. Hay un programa
    de conversión que convierte gráficos de |xfig| a \tikzname.
\end{enumerate}


\subsection{Utility Packages}

The \pgfname\ package comes along with a number of utility package that are not
really about creating graphics and which can be used independently of \pgfname.
However, they are bundled with \pgfname, partly out of convenience, partly
because their functionality is closely intertwined with \pgfname. These utility
packages are:
%
\begin{enumerate}
    \item The |pgfkeys| package defines a powerful key management facility.
        It can be used completely independently of \pgfname.
    \item The |pgffor| package defines a useful |\foreach| statement.
    \item The |pgfcalendar| package defines macros for creating calendars.
        Typically, these calendars will be rendered using \pgfname's graphic
        engine, but you can use |pgfcalendar| also typeset calendars using
        normal text. The package also defines commands for ``working'' with
        dates.
    \item The |pgfpages| package is used to assemble several pages into a
        single page. It provides commands for assembling several ``virtual
        pages'' into a single ``physical page''. The idea is that whenever
        \TeX\ has a page ready for ``shipout'', |pgfpages| interrupts this
        shipout and instead stores the page to be shipped out in a special
        box. When enough ``virtual pages'' have been accumulated in this way,
        they are scaled down and arranged on a ``physical page'', which then
        \emph{really} shipped out. This mechanism allows you to create ``two
        page on one page'' versions of a document directly inside \LaTeX\
        without the use of any external programs. However, |pgfpages| can do
        quite a lot more than that. You can use it to put logos and watermark
        on pages, print up to 16 pages on one page, add borders to pages, and
        more.
\end{enumerate}


\subsection{How to Read This Manual}

This manual describes both the design of \tikzname\ and its usage. The
organization is very roughly according to ``user-friendliness''. The commands
and subpackages that are easiest and most frequently used are described first,
more low-level and esoteric features are discussed later.

If you have not yet installed \tikzname, please read the installation first.
Second, it might be a good idea to read the tutorial. Finally, you might wish
to skim through the description of \tikzname. Typically, you will not need to
read the sections on the basic layer. You will only need to read the part on
the system layer if you intend to write your own frontend or if you wish to
port \pgfname\ to a new driver.

The ``public'' commands and environments provided by the system are described
throughout the text. In each such description, the described command,
environment or option is printed in red. Text shown in green is optional and
can be left out.


\subsection{Authors and Acknowledgements}
\label{section-authors}

The bulk of the \pgfname\ system and its documentation was written by Till
Tantau. A further member of the main team is Mark Wibrow, who is responsible,
for example, for the \pgfname\ mathematical engine, many shapes, the decoration
engine, and matrices. The third member is Christian Feuers\"anger who
contributed the floating point library, image externalization, extended key
processing, and automatic hyperlinks in the manual.

Furthermore, occasional contributions have been made by Christophe Jorssen,
Jin-Hwan Cho, Olivier Binda, Matthias Schulz, Ren\'ee Ahrens, Stephan Schuster,
and Thomas Neumann.

Additionally, numerous people have contributed to the \pgfname\ system by
writing emails, spotting bugs, or sending libraries and patches. Many thanks to
all these people, who are too numerous to name them all!


\subsection{Getting Help}

When you need help with \pgfname\ and \tikzname, please do the following:

\begin{enumerate}
    \item Read the manual, at least the part that has to do with your
        problem.
    \item If that does not solve the problem, try having a look at the
        GitHub development page for \pgfname\ and \tikzname\ (see the
        title of this document). Perhaps someone has already reported a
        similar problem and someone has found a solution.
    \item On the website you will find numerous forums for getting help.
        There, you can write to help forums, file bug reports, join mailing
        lists, and so on.
    \item Before you file a bug report, especially a bug report concerning
        the installation, make sure that this is really a bug. In particular,
        have a look at the |.log| file that results when you \TeX\ your
        files. This |.log| file should show that all the right files are
        loaded from the right directories. Nearly all installation problems
        can be resolved by looking at the |.log| file.
    \item \emph{As a last resort} you can try to email me (Till Tantau) or,
        if the problem concerns the mathematical engine, Mark Wibrow. I do
        not mind getting emails, I simply get way too many of them. Because
        of this, I cannot guarantee that your emails will be answered in a 
        timely fashion or even at all. Your chances that your problem will
        be fixed are somewhat higher if you mail to the \pgfname\ mailing
        list (naturally, I read this list and answer questions when I have
        the time).
\end{enumerate}
