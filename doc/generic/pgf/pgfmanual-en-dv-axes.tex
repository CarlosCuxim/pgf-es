% Copyright 2019 by Till Tantau
%
% This file may be distributed and/or modified
%
% 1. under the LaTeX Project Public License and/or
% 2. under the GNU Free Documentation License.
%
% See the file doc/generic/pgf/licenses/LICENSE for more details.


\section{Axes}
\label{section-dv-axes}

\subsection{Overview}

When a data point is visualized, the most obvious way of creating a visual
representation of its many attributes is to vary \emph{where} the data point is
shown. The data visualization system uses \emph{axes} to turn data point
attributes into positions on a page. The simplest -- and most common -- use of
axes is to vary the horizontal position of data points according to one
attribute and to vary the vertical position according to another attribute. In
contrast, in a polar plot one attribute dictates the distance of the data point
from the origin and another attribute describes the angle. From the data
visualization engine's point of view, in both cases two \emph{axes} are
involved.

In addition to specifying how the value of a certain attribute is converted
into a displacement on the page, an axis is also typically (but not always)
visualized (``drawn'') somewhere on the page. In this case, it is also
customary to add a visual representation on this axis of which attribute values
correspond to which positions on the page -- something commonly known as
\emph{ticks}. Similar to ticks, \emph{grid lines} also indicate positions where
a certain attribute has a certain value, but instead of just indicating a
single position on an axis, a grid line goes through all points that share an
attribute value.

In the following, in Section~\ref{section-dv-axes-main} we first have a look at
how axes can be defined and configured. As you will see, a lot of powerful
configurations are available, but you will rarely define and configure an axis
from scratch. Rather, it is more common to use a preconfigured axis instead.
Section~\ref{section-dv-axis-systems} introduces \emph{axis systems}, which are
predefined bundles of axes. You can define your own axis systems, but, again,
in most cases it will suffice to just use one of the many preconfigured axis
systems and use a few options to configure it so that it fits your need.
Section~\ref{section-dv-ticks-and-grids} explains how ticks and grid lines can
be configured. Again, several layers of options allow you to configure the way
ticks look and where they are placed in great detail.

This section documents the standard axis systems that are always available. For
polar axis systems, a special library needs to be loaded, which is documented
in Section~\ref{section-dv-polar}.


\subsection{Basic Configuration of Axes}
\label{section-dv-axes-main}

Inside the data visualization system, an \emph{axis} is roughly a ``systematic,
named way of mapping an attribute to a position on a page''. For instance, the
classical ``$x$-axis'' is the ``systematic way of mapping the value of the |x|
attribute of data points to a horizontal position on the page''. An axis is
\emph{not} its visual representation (such as the horizontal line with the
ticks drawn to represent the $x$-axis), but a visual representation can be
created once an axis has been defined.

The transformation of an attribute value (such as the value |1000000000| for
the |x| attribute) to a specific displacement of the corresponding data point
on the page involves two steps:
%
\begin{enumerate}
    \item First, the range of possible values such as $[-5.6\cdot
        10^{12},7.8\cdot 10^{12}]$ must be mapped to a ``reasonable'' interval
        such as $[0\mathrm{cm},5\mathrm{cm}]$ or $[0^\circ,180^\circ]$.
        \tikzname's drawing routines will only be able to cope with values from
        such a ``reasonable'' interval.
    \item Second, the values from the reasonable interval must be mapped to a
        transformation.
\end{enumerate}
%
The first step is always the same for all axes, while the second requires
different strategies. For this reason, the command |new axis base| is used to
create a ``basic'' axis that has a ``scaling mapper'', whose job it is to map
the range of values of a specific attribute to a reasonable interval, but such
a basic axis does not define an actual transformation object. For this second
step, additional objects such as a |linear transformer| need to be created
separately.


\subsubsection{Usage}

To create an axis, the key |new axis base| is used first. Since this key does
not create a transformation object, users typically do not use this key
directly. Rather, it is used internally by other keys that create ``real''
axes. These keys are listed in Section~\ref{section-dv-reference-axis-types}.

\begin{key}{/tikz/data visualization/new axis base=\meta{axis name}}
    This key defines a new axis for the current data visualization called
    \meta{name}. This has two effects:
    %
    \begin{enumerate}
        \item A so called \emph{scaling mapper} is created that will monitor a
            certain attribute, rescale it, and map it to another attribute.
            (This will be explained in detail in a moment.)
        \item The \meta{axis name} is made available as a key that can be used
            to configure the axis:
            %
            \begin{key}{/tikz/data visualization/\meta{axis name}=\meta{options}}
                This key becomes available once |new axis base=|meta{axis name}
                has been called. It will execute the \meta{options} with the
                path prefix |/tikz/data visualization/axis options|.
                %
\begin{codeexample}[code only]
[new axis base=my axis,
 my axis={attribute=some attribute}]
\end{codeexample}
            \end{key}
        \item The \meta{axis name} becomes part of the current set of axes.
            This set can be accessed through the following key:
            %
            \begin{key}{/tikz/data visualization/all axes=\meta{options}}
                This key passes the \meta{options} to all axes inside the
                current scope, just as if you had written \meta{some axis
                name}|=|\meta{options} for each \meta{some axis name} in the
                current scope, including the just-created name \meta{axis
                name}.
            \end{key}
    \end{enumerate}
    %
    There are many \meta{options} that can be passed to a newly created axis.
    They are explained in the rest of this section.
\end{key}

Note the |new axis base| does \emph{not} cause attributes to be mapped to
positions on a page. Rather, special keys like |new Cartesian axis| first use
|new axis base| to create an axis and then create an internal object that
performs a linear mapping of the attribute to positions along a vectors.


\subsubsection{The Axis Attribute}
\label{section-dv-axis-attribute}

The first main job of an axis is to map the different values of some attribute
to a reasonable interval. To achieve this, the following options are important
(recall that these options are passed to the key whose name is the name of the
axis):

\begin{key}{/tikz/data visualization/axis options/attribute=\meta{attribute}}
    Specifies that the axis is used to transform the data points according the
    different values of the key |/data point/|\meta{attribute}. For instance,
    when we create a classical two-dimensional Cartesian coordinate system,
    then there are two axes called |x axis| and |y axis| that monitor the
    values of the attributes |/data point/x| and |/data point/y|, respectively:
    %
\begin{codeexample}[code only]
  [new axis base=x axis,
   new axis base=y axis,
   x axis={attribute=x},
   y axis={attribute=y}]
\end{codeexample}
    %
    In another example, we also create an |x axis| and a |y axis|. However,
    this time, we want to plot the values of the |/data point/time| attribute
    on the $x$-axis and, say, the value of the |height| attribute on the
    $y$-axis:
    %
\begin{codeexample}[code only]
  [new axis base=x axis,
   new axis base=y axis,
   x axis={attribute=time},
   y axis={attribute=height}]
\end{codeexample}
    %
    During the data visualization, the \meta{attribute} will be ``monitored''
    during the survey phase. This means that for each data point, the current
    value of |/data point/|\meta{attribute} is examined and the minimum value
    of all of these values as well as the maximum value is recorded internally.
    Note that this works even when very large numbers like |100000000000| are
    involved.

    Here is a real-life example. The |scientific axes| create two axes, called
    |x axis| and |y axis|, respectively.
    %
\begin{codeexample}[preamble={\usetikzlibrary{datavisualization}}]
\tikz \datavisualization [scientific axes,
                          x axis={attribute=people, length=2.5cm, ticks=few},
                          y axis={attribute=year},
                          visualize as scatter]
  data {
    year, people
    1900, 100
    1910, 200
    1950, 200
    1960, 250
    2000, 150
  };
\end{codeexample}
    %
\end{key}


\subsubsection{The Axis Attribute Range Interval}

Once an attribute has been specified for an axis, the data visualization engine
will start monitoring this value. This means that before anything actual
visualization is done, a ``survey phase'' is used to determine the range of
values encountered for the attribute for all data points. This range of values
results in what is called the \emph{attribute range interval}. Its minimum is
the smallest value encountered in the data and its maximum is the largest
value.

Even though the attribute range interval is computed automatically and even
though you typically do not need to worry about it, there are some situations
where you may wish to set or enlarge the attribute range interval:
%
\begin{itemize}
    \item You may wish to start the interval with $0$, even though the range of
        values contains only positive values.
    \item You may wish to slightly enlarge the interval so that, say, the
        maximum is some ``nice'' value like |100| or |60|.
\end{itemize}

The following keys can be used to influence the size of the attribute range
interval:
%
\begin{key}{/tikz/data visualization/axis options/include value=\meta{list of value}}
    This key ``fakes'' data points for which the attribute's values are in the
    comma-separated \meta{list of values}. For instance, when you write
    |include value=0|, then the attribute range interval is guaranteed to
    contain |0| -- even if the actual data points are all positive or all
    negative.
    %
\begin{codeexample}[preamble={\usetikzlibrary{datavisualization.formats.functions}}]
\tikz \datavisualization [scientific axes, all axes={length=3cm},
                          visualize as line]
  data [format=function] {
    var x : interval [5:10];
    func y = \value x * \value x;
  };
\end{codeexample}
    %
\begin{codeexample}[preamble={\usetikzlibrary{datavisualization.formats.functions}}]
\tikz \datavisualization [scientific axes, all axes={length=3cm},
                          visualize as line,
                          x axis={include value=20},
                          y axis={include value=0}]
  data [format=function] {
    var x : interval [5:10];
    func y = \value x * \value x;
  };
\end{codeexample}
    %
\end{key}

\begin{key}{/tikz/data visualization/axis options/min value=\meta{value}}
    This key allows you to simply set the minimum value, regardless of which
    values are present in the actual data. This key should be used with care:
    If there are data points for which the attribute's value is less than
    \meta{value}, they will still be depicted, but typically outside the normal
    visualization area. Usually, saying |include value=|\meta{value} will
    achieve the same as saying |min value=|\meta{value}, but with less danger
    of creating ill-formed visualizations.
\end{key}

\begin{key}{/tikz/data visualization/axis options/max value=\meta{value}}
    Works like |min value|.
\end{key}


\subsubsection{Scaling: The General Mechanism}

The above key allows us specify which attribute should be ``monitored''. The
next key is used to specify what should happen with the observed values.

\begin{key}{/tikz/data visualization/axis options/scaling=\meta{scaling spec}}
    The \meta{scaling spec} must have the following form:
    %
    \begin{quote}
        \meta{$s_1$}| at |\meta{$t_1$}| and |\meta{$s_2$}| at |\meta{$t_2$}
    \end{quote}
    %
    This means that monitored values in the interval $[s_1,s_2]$ should be
    mapped to values the ``reasonable'' interval $[t_1,t_2]$, instead. For
    instance, we might write
    %
\begin{codeexample}[code only]
[y axis = {scaling = 1900 at 0cm and 2000 at 5cm}]
\end{codeexample}
    %
    in order to map dates between 1900 and 2000 to the dimension interval
    $[0\mathrm{cm},5\mathrm{cm}]$.
    %
\begin{codeexample}[preamble={\usetikzlibrary{datavisualization}}]
\tikz \datavisualization
   [scientific axes,
    x axis={attribute=people, length=2.5cm, ticks=few},
    y axis={attribute=year, scaling=1900 at 0cm and 2000 at 5cm},
    visualize as scatter]
  data {
    year, people
    1900, 100
    1910, 200
    1950, 200
    1960, 250
    2000, 150
  };
\end{codeexample}
    %
    So much for the basic idea. Let us now have a detailed look at what
    happens.


    \medskip
    \textbf{Number format and the min and max keywords.}
    The source values $s_1$ and $s_2$ are typically just numbers like |3.14| or
    |10000000000|. However, as described in
    Section~\ref{section-dv-expressions}, you can also specify expressions like
    |(pi/2)|, provided that (currently) you put them in parentheses.

    Instead of a number, you may alternatively also use the two key words |min|
    and |max| for $s_1$ and/or $s_2$. In this case, |min| evaluates to the
    smallest value observed for the attribute in the data, symmetrically |max|
    evaluates to the largest values. For instance, in the above example with
    the |year| attribute ranging from |1900| to |2000|, the keyword |min| would
    stand for |1900| and |max| for |2000|. Similarly, for the |people|
    attribute |min| stands for |100| and |max| for |250|. Note that |min| and
    |max| can only be used for $s_1$ and $s_2$, not for $t_1$ and $t_2$.

    A typical use of the |min| and |max| keywords is to say
    %
\begin{codeexample}[code only]
scaling = min at 0cm and max at 5cm
\end{codeexample}
    %
    to map the complete range of values into an interval of length of 5cm.

    The interval  $[s_1,s_2]$ need not contain all values that the
    \meta{attribute} may attain. It is permissible that values are less than
    $s_1$ or more than $s_2$.


    \medskip
    \textbf{Linear transformation of the attribute.}
    As indicated earlier, the main job of an axis is to map values from a
    ``large'' interval $[s_1,s_2]$ to a more reasonable interval $[t_1,t_2]$.
    Suppose that for the current data point the value of the key
    |/data point/|\meta{attribute} is the number $v$. In the simplest case, the
    following happens: A new value $v'$ is computed so that $v' = t_1$ when
    $v=s_1$ and $v'=t_2$ when $v=s_2$ and $v'$ is some value in between $t_1$
    and $t_2$ then $v$ is some value in between $s_1$ and $s_2$. (Formally, in
    this basic case $v' = t_1 + (v-s_1)\frac{t_2-t_1}{s_2-s_1}$.)

    Once $v'$ has been computed, it is stored in the key
    |/data point/|\meta{attribute}|/scaled|. Thus, the ``reasonable'' value
    $v'$ does not replace the value of the attribute, but it is placed in a
    different key. This means that both the original value and the more
    ``scaled'' values are available when the data point is visualized.

    As an example, suppose you have written
    %
\begin{codeexample}[code only]
[x axis = {attribute = x, scaling=1000 at 20 and 2000 at 30}]
\end{codeexample}
    %
    Now suppose that |/data point/x| equals |1200| for a data point. Then the
    key |/data point/x/scaled| will be set to |22| when the data point is being
    visualized.


    \medskip
    \textbf{Nonlinear transformations of the attribute.}
    By default, the transformation of $[s_1,s_2]$ to $[t_1,t_2]$ is the linear
    transformation described above. However, in some case you may be interested
    in a different kind of transformation: For example, in a logarithmic plot,
    values of an attribute may range between, say, |1| and |1000| and we want
    an axis of length |3cm|. So, we would write
    %
\begin{codeexample}[code only]
[x axis = {attribute = x, scaling=1 at 0cm and 1000 at 3cm}]
\end{codeexample}
    %
    Indeed, |1| will now be mapped to position |0cm| and |1000| will be mapped
    to position |3cm|. Now, the value |10| will be mapped to approximately
    |0.03cm| because it is (almost) at one percent between |1| and |1000|.
    However, in a logarithmic plot we actually want |10| to be mapped to the
    position |1cm| rather than |0.03cm| and we want |100| to be mapped to the
    position |2cm|. Such a mapping a \emph{nonlinear} mapping between the
    intervals.

    In order to achieve such a nonlinear mapping, the |function| key can be
    used, whose syntax is described in a moment. The effect of this key is to
    specify a function $f \colon \mathbb{R} \to \mathbb{R}$ like, say, the
    logarithm function. When such a function is specified, the mapping of $v$
    to $v'$ is computed as follows:
    %
    \begin{align*}
        v' = t_1 + (f(s_2) - f(v))\frac{t_2 - t_1}{f(s_2)-f(s_1)}.
    \end{align*}

    The syntax of the |function| key is described next, but you typically will
    not call this key directly. Rather, you will use a key like |logarithmic|
    that installs appropriate code for the |function| key for you.
    %
    \begin{key}{/tikz/data visualization/axis options/function=\meta{code}}
        The \meta{code} should specify a function $f$ that is applied during
        the transformation of the interval $[s_1,s_2]$ to the interval
        $[t_1,t_2]$ in the following way: When the \meta{code} is called, the
        macro |\pgfvalue| will have been set to an internal representation of
        the to-be-transformed value~$v$. You can then call the commands of the
        math-micro-kernel of the data visualization system, see
        Section~\ref{section-dv-math-kernel}, to compute a new value. This new
        value must once more be stored in |\pgfvalue|.

        The most common use of this key is to say
        %
\begin{codeexample}[code only]
some axis={function=\pgfdvmathln{\pgfvalue}{\pgfvalue}}
\end{codeexample}
        %
        This specifies that the function $f$ is the logarithm function.
        %
\begin{codeexample}[preamble={\usetikzlibrary{datavisualization}}]
\tikz \datavisualization
   [scientific axes,
    x axis={ticks={major={at={1,10,100,1000}}},
             scaling=1 at 0cm and 1000 at 3cm,
             function=\pgfdvmathln{\pgfvalue}{\pgfvalue}},
    visualize as scatter]
  data [format=named] {
    x={1,100,...,1000}, y={1,2,3}
  };
\end{codeexample}
        %
        Another possibility might be to use the square-root function, instead:
        %
\begin{codeexample}[preamble={\usetikzlibrary{datavisualization}}]
\tikz \datavisualization
   [scientific axes,
    x axis={ticks=few,
            scaling=1 at 0cm and 1000 at 3cm,
            function=\pgfdvmathunaryop{\pgfvalue}{sqrt}{\pgfvalue}},
    visualize as scatter]
  data [format=named] {
    x={0,100,...,1000}, y={1,2,3}
  };
\end{codeexample}
    \end{key}


    \medskip
    \textbf{Default scaling.}
    When no scaling is specified, it may seem natural to use $[0,1]$ both as
    the source and the target interval. However, this would not work when the
    logarithm function is used as transformations: In this case the logarithm
    of zero would be computed, leading to an error. Indeed, for a logarithmic
    axis it is far more natural to use $[1,10]$ as the source interval and
    $[0,1]$ as the target interval.

    For these reasons, the default value for the |scaling| that is used when no
    value is specified explicitly can be set using a special key:
    %
    \begin{key}{/tikz/data visualization/axis options/scaling/default=\meta{text}}
        The \meta{text} is used as |scaling| whenever no other scaling is
        specified. This key is mainly used when a transformation function is
        set using |function|; normally, you will not use this key directly.
    \end{key}
\end{key}

Most of the time, you will not use neither the |scaling| nor the |function| key
directly, but rather you will use one of the following predefined styles
documented in the following.


\subsubsection{Scaling: Logarithmic Axes}

\begin{key}{/tikz/data visualization/axis options/logarithmic}
    When this key is used with an axis, three things happen:
    %
    \begin{enumerate}
        \item The transformation |function| of the axis is setup to the
            logarithm.
        \item The strategy for automatically generating ticks and grid lines is
            set to the |exponential strategy|, see
            Section~\ref{section-dv-exponential-strategy} for details.
        \item The default scaling is setup sensibly.
    \end{enumerate}
    %
    All told, to turn an axis into a logarithmic axis, you just need to add
    this option to the axis.
    %
\begin{codeexample}[
    width=8cm,
    preamble={\usetikzlibrary{datavisualization.formats.functions}},
]
\tikz \datavisualization [scientific axes,
                          x axis={logarithmic},
                          y axis={logarithmic},
                          visualize as line]
 data [format=function] {
   var x : interval [0.01:100];
   func y = \value x * \value x;
 };
\end{codeexample}
    %
    Note that this will work with any axis, including, say, the degrees on a
    polar axis:
    %
\begin{codeexample}[preamble={\usetikzlibrary{datavisualization.polar}}]
\tikz \datavisualization
    [new polar axes,
     angle axis={logarithmic, scaling=1 at 0 and 90 at 90},
     radius axis={scaling=0 at 0cm and 100 at 3cm},
     visualize as scatter]
  data [format=named] {
    angle={1,10,...,90}, radius={1,10,...,100}
  };
\end{codeexample}
    %
\begin{codeexample}[preamble={\usetikzlibrary{datavisualization.polar}}]
\tikz \datavisualization
    [new polar axes,
     angle axis={degrees},
     radius axis={logarithmic, scaling=1 at 0cm and 100 at 3cm},
     visualize as scatter]
  data [format=named] {
    angle={1,10,...,90}, radius={1,10,...,100}
  };
\end{codeexample}
    %
\end{key}


\subsubsection{Scaling: Setting the Length or Unit Length}

\begin{key}{/tikz/data visualization/axis options/length=\meta{dimension}}
    Sets |scaling| to |min at 0cm and max at |\meta{dimension}. The effect is
    that the range of all values of the axis's attribute will be mapped to an
    interval of exact length \meta{dimension}.
    %
\begin{codeexample}[preamble={\usetikzlibrary{datavisualization}}]
\tikz \datavisualization [scientific axes,
                          x axis={length=3cm},
                          y axis={length=2cm},
                          all axes={ticks=few},
                          visualize as line]
    data {
      x, y
      10, 10
      20, 20
      15, 30
      13, 20
    };
\end{codeexample}
    %
\begin{codeexample}[preamble={\usetikzlibrary{datavisualization}}]
\tikz \datavisualization [scientific axes,
                          x axis={length=3cm},
                          y axis={length=4cm},
                          all axes={ticks=few},
                          visualize as line]
    data {
      x, y
      10, 10
      20, 20
      15, 30
      13, 20
    };
\end{codeexample}
    %
\end{key}

\begin{key}{/tikz/data visualization/axis options/unit length=\meta{dimension}\opt{| per |\meta{number}| units|}}
    Sets |scaling| to |0 at 0cm and 1 at |\meta{dimension}. In other words,
    this key allows you to specify how long a single unit should be. This key
    is particularly useful when you wish to ensure that the same scaling is
    used across multiple axes or pictures.
    %
\begin{codeexample}[preamble={\usetikzlibrary{datavisualization}}]
\tikz \datavisualization [scientific axes,
                          all axes={ticks=few, unit length=1mm},
                          visualize as line]
    data {
      x, y
      10, 10
      40, 20
      15, 30
      13, 20
    };
\end{codeexample}
    %
    The optional |per |\meta{number}| units| allows you to apply more drastic
    scaling. Suppose that you want to plot a graph where one billion
    corresponds to one centimeter. Then the unit length would be need to be set
    to a hundredth of a nanometer -- much too small for \TeX\ to handle as a
    dimension. In this case, you can write
    |unit length=1cm per 1000000000 units|:
    %
\begin{codeexample}[preamble={\usetikzlibrary{datavisualization}}]
\tikz \datavisualization
  [scientific axes,
   x axis={unit length=1mm per 1000000000 units, ticks=few},
   visualize as line]
 data {
   x, y
   10000000000, 10
   40000000000, 20
   15000000000, 30
   13000000000, 20
 };
\end{codeexample}
    %
\end{key}
%
\begin{key}{/tikz/data visualization/axis options/power unit length=\meta{dimension}}
    This key is used in conjunction with the |logarithmic| setting. It cases
    the |scaling| to be set to |1 at 0cm and 10 at |\meta{dimension}. This
    causes a ``power unit'', that is, one power of ten in a logarithmic plot,
    to get a length of \meta{dimension}. Again, this key is useful for ensuring
    that the same scaling is used across multiple axes or pictures.
    %
\begin{codeexample}[width=8cm,preamble={\usetikzlibrary{datavisualization}}]
\tikz \datavisualization
  [scientific axes,
   y axis={logarithmic, power unit length=1mm, grid},
   visualize as line]
 data {
   x, y
   0, 0.0000000001
   1, 1
   2, 100000
   3, 100000000000
   4, 10000000000000000000000000000000
   5, 500000000
   6, 5000000000000000000
 };
\end{codeexample}
    %
\end{key}


\subsubsection{Axis Label}

An axis can have a \emph{label}, which is a textual representation of the
attribute according to which the axis varies the position of the page. You can
set the attribute using the following key:

\begin{key}{/tikz/data visualization/axis options/label=\opt{|\char`\{[|\meta{options}|]|}\meta{text}\opt{|\char`\}|}
        (default \normalfont axis's label in math mode)%
}
    This key sets the label of an axis to \meta{text}. This text will typically
    be placed inside a |node| and the \meta{options} can be used to further
    configure the way this node is rendered. The \meta{options} will be
    executed with the path prefix |/tikz/data visualization/|, so you need to
    say |node style| to configure the styling of a node, see
    Section~\ref{section-dv-style}.
    %
\begin{codeexample}[preamble={\usetikzlibrary{datavisualization.formats.functions}}]
\tikz \datavisualization [
    scientific axes,
    x axis = {label, length=2.5cm},
    y axis = {label={[node style={fill=blue!20}]{$x^2$}}},
    visualize as smooth line]
 data [format=function] {
   var x : interval [-3:5];
   func y = \value x * \value x;
 };
\end{codeexample}
    %
\end{key}

Note that using the |label| key does not actually cause a node to be created,
because it is somewhat unclear where the label should be placed. Instead, the
|visualize label| key is used (typically internally by an axis system) to show
the label at some sensible position. This key is documented in
Section~\ref{section-dv-visualize-label}.


\subsubsection{Reference: Axis Types}
\label{section-dv-reference-axis-types}

As explained earlier, when you use |new axis base| to create a new axis, a
powerful scaling and attribute mapping mechanism is installed, but no mapping
of values to positions on the page is performed. For this, a
\emph{transformation object} must be installed. The following keys take care of
this for you. Note, however, that even these keys do not cause a visual
representation of the axis to be added to the visualization -- this is the job
of an axis system, see Section~\ref{section-dv-axis-systems}.

\begin{key}{/tikz/data visualization/new Cartesian axis=\meta{name}}
    This key creates a new ``Cartesian'' axis, named \meta{name}. For such an
    axis, the (scaled) values of the axis's attribute are transformed into a
    displacement on the page along a straight line. The following key is used
    to configure in which ``direction'' the axis points:
    %
    \begin{key}{/tikz/data visualization/axis options/unit vector=\meta{coordinate} (initially {(1pt,0pt)})}
        Recall that an axis takes the values of an attribute and rescales them
        so that they fit into a ``reasonable'' interval $[t_1,t_2]$. Suppose
        that $v'$ is the rescaled dimension in (\TeX) points. Then when the
        data point is visualized, the coordinate system will be shifted by $v'$
        times the \meta{coordinate}.

        As an example, suppose that you have said
        |scaling=0 and 10pt and 50 and 20pt|. Then when the underlying
        attribute has the value |25|, it will be mapped to a $v'$ of $15$
        (because |25| lies in the middle of |0| and |50| and |15pt| lies in the
        middle of |10pt| and |20pt|). This, in turn, causes the data point to
        be displaced by $15$ times the \meta{coordinate}.

        The bottom line is that the \meta{coordinate} should usually denote a
        point that is at distance |1pt| from the origin and that points into
        the direction of the axis.
        %
\begin{codeexample}[preamble={\usetikzlibrary{datavisualization}}]
\begin{tikzpicture}
  \draw [help lines] (0,0) grid (3,2);

  \datavisualization
    [new Cartesian axis=x axis, x axis={attribute=x},
     new Cartesian axis=y axis, y axis={attribute=y},
     x axis={unit vector=(0:1pt)},
     y axis={unit vector=(60:1pt)},
     visualize as scatter]
  data {
    x, y
    0, 0
    1, 0
    2, 0
    1, 1
    2, 1
    1, 1.5
    2, 1.5
  };
\end{tikzpicture}
\end{codeexample}
    \end{key}
\end{key}


\subsection{Axis Systems}
\label{section-dv-axis-systems}

An \emph{axis system} is, as the name suggests, a whole family of axes that act
in concert. For example, in the ``standard'' axis system there is a horizontal
axis called the $x$-axis that monitors the |x| attribute (by default, you can
change this easily) and a vertical axis called the $y$-axis. Furthermore, a
certain number of ticks are added and labels are placed at sensible positions.


\subsubsection{Usage}

Using an axis system is usually pretty easy: You just specify a key like
|scientific axes| and the necessary axes get initialized with sensible default
values. You can then start to modify these default values, if necessary.

First, you can (and should) set the attributes to which the difference axes
refer. For instance, if the |time| attribute is plotted along the $x$-axis, you
would write
%
\begin{codeexample}[code only]
x axis = {attribute = time}
\end{codeexample}

Second, you may wish to modify the lengths of the axes. For this, you can use
keys like |length| or further keys as described in the references later on.

Third, you may often wish to modify how many ticks and grid lines are shown. By
default, no grid lines are shown, but you can say the following in order to
cause grid lines to be shown:
%
\begin{codeexample}[code only]
all axes={grid}
\end{codeexample}
%
Naturally, instead of |all axes| you can also specify a single axis, causing
only grid lines to be shown for this axis. In order to change the number of
ticks that are shown, you can say
%
\begin{codeexample}[code only]
all axes={ticks=few}
\end{codeexample}
%
or also |many| instead of |few| or even |none|. Far more fine-grained control
over the tick placement and rendering is possible, see
Section~\ref{section-dv-ticks-and-grids} for details.

Fourth, consider adding units (like ``cm'' for centimeters or
``$\mathrm{m}/\mathrm{s}^2$'' for acceleration) to your ticks:
%
\begin{codeexample}[code only]
x axis={ticks={tick unit=cm}}, y axis={ticks={tick unit=m/s^2}}
\end{codeexample}

Finally, consider adding labels to your axes. For this, use the label option:
%
\begin{codeexample}[code only]
x axes={time $t$ (ms)}, y axis={distance $d$ (mm)}
\end{codeexample}

Here is an example that employs most of the above features:
%
\begin{codeexample}[width=8.5cm,preamble={\usetikzlibrary{datavisualization}}]
\tikz \datavisualization [
  scientific axes=clean,
  x axis={attribute=time, ticks={tick unit=ms},
    label={elapsed time}},
  y axis={attribute=v, ticks={tick unit=m/s},
    label={speed of disc}},
  all axes=grid,
  visualize as line]
data {
  time, v
  0, 0
  1, 0.001
  2, 0.002
  3, 0.004
  4, 0.0035
  5, 0.0085
  6, 0.0135
};
\end{codeexample}


\subsubsection{Reference: Scientific Axis Systems}

\begin{key}{/tikz/data visualization/scientific axes=\opt{\meta{options}}}
    This key installs a two-dimensional coordinate system based on the
    attributes |/data point/x| and |/data point/y|.
    %
\begin{codeexample}[
    width=7cm,
    preamble={\usetikzlibrary{datavisualization.formats.functions}},
]
\begin{tikzpicture}
  \datavisualization [scientific axes,
                      visualize as smooth line]
    data [format=function] {
      var x : interval [0:100];
      func y = sqrt(\value x);
    };
\end{tikzpicture}
\end{codeexample}

    This axis system is usually a good choice to depict ``arbitrary two
    dimensional data''. Because the axes are automatically scaled, you do not
    need to worry about how large or small the values will be. The name
    |scientific axes| is intended to indicate that this axis system is often
    used in scientific publications.

    You can use the \meta{options} to fine tune the axis system. The
    \meta{options} will be executed with the following path prefix:
    %
\begin{codeexample}[code only]
/tikz/data visualization/scientific axes
\end{codeexample}
    %
    All keys with this prefix can thus be passed as \meta{options}.

    This axis system will always distort the relative magnitudes of the units
    on the two axis. If you wish the units on both axes to be equal, consider
    directly specifying the unit length ``by hand'':
    %
\begin{codeexample}[preamble={\usetikzlibrary{datavisualization.formats.functions}}]
\begin{tikzpicture}
  \datavisualization [visualize as smooth line,
                      scientific axes,
                      all axes={unit length=1cm per 10 units, ticks={few}}]
    data [format=function] {
      var x : interval [0:100];
      func y = sqrt(\value x);
    };
\end{tikzpicture}
\end{codeexample}

    The |scientific axes| have the following properties:
    %
    \begin{itemize}
        \item The |x|-values are surveyed and the $x$-axis is then scaled and
            shifted so  that it has the length specified by the following key.
            %
            \begin{key}{/tikz/data visualization/scientific axes/width=\meta{dimension} (initially 5cm)}
            \end{key}
            %
            The minimum value is at the left end of the axis and at the canvas
            origin. The maximum value is at the right end of the axis. \item
            The |y|-values are surveyed and the $y$-axis is then scaled so that
            is has the length specified by the following key.
            %
            \begin{key}{/tikz/data visualization/scientific axes/height=\meta{dimension}}
                By default, the |height| is the golden ratio times the |width|.
            \end{key}
            %
            The minimum value is at the bottom of the axis and at the canvas
            origin. The maximum value is at the top of the axis.
        \item Lines (forming a frame) are depicted at the minimum and maximum
            values of the axes in 50\% black.
    \end{itemize}

    The following keys are executed by default as options: |outer ticks| and
    |standard labels|.

    You can use the following style to overrule the defaults:

    \begin{stylekey}{/tikz/data visualization/every scientific axes}
    \end{stylekey}
\end{key}

The keys described in the following can be used to fine-tune the way the
scientific axis system is rendered.

\begin{key}{/tikz/data visualization/scientific axes/outer ticks}
    This causes the ticks to be drawn `` on the outside'' of the frame so that
    they interfere as little as possible with the data. It is the default.
    %
\begin{codeexample}[
    width=7cm,
    preamble={\usetikzlibrary{datavisualization.formats.functions}},
]
\begin{tikzpicture}
  \datavisualization [scientific axes=outer ticks,
                      visualize as smooth line]
    data [format=function] {
      var x : interval [-12:12];
      func y = \value x*\value x*\value x;
    };
\end{tikzpicture}
\end{codeexample}
    %
\end{key}

\begin{key}{/tikz/data visualization/scientific axes/inner ticks}
    This axis system works like |scientific axes|, only the ticks are on the
    ``inside'' of the frame.
    %
\begin{codeexample}[
    width=7cm,
    preamble={\usetikzlibrary{datavisualization.formats.functions}},
]
\begin{tikzpicture}
  \datavisualization [scientific axes=inner ticks,
                      visualize as smooth line]
    data [format=function] {
      var x : interval [-12:12];
      func y = \value x*\value x*\value x;
    };
\end{tikzpicture}
\end{codeexample}

    This axis system is also common in publications, but the ticks tend to
    interfere with marks if they are near to the border as can be seen in the
    following example:
    %
\begin{codeexample}[preamble={\usetikzlibrary{datavisualization}}]
\begin{tikzpicture}
  \datavisualization [scientific axes={inner ticks, width=3.2cm},
                      style sheet=cross marks,
                      visualize as scatter/.list={a,b}]
    data [set=a] {
      x, y
      0, 0
      1, 1
      0.5, 0.5
      2, 1
    }
    data [set=b] {
      x, y
      0.05, 0
      1.5, 1
      0.5, 0.75
      2, 0.5
    };
\end{tikzpicture}
\end{codeexample}
    %
\end{key}

\begin{key}{/tikz/data visualization/scientific axes/clean}
    The axes and the ticks are completely removed from the actual data, making
    this axis system especially useful for scatter plots, but also for most
    other scientific plots.
    %
\begin{codeexample}[
    width=7.5cm,
    preamble={\usetikzlibrary{datavisualization.formats.functions}},
]
\tikz \datavisualization [
  scientific axes=clean,
  visualize as smooth line]
data [format=function] {
  var x : interval [-12:12];
  func y = \value x*\value x*\value x;
};
\end{codeexample}

    The distance of the axes from the actual plot is given by the padding of
    the axes.
\end{key}

For all scientific axis systems, different label placement strategies can be
specified. They are discussed in the following.

\begin{key}{/tikz/data visualization/scientific axes/standard labels}
    As the name suggests, this is the standard placement strategy. The label of
    the $x$-axis is placed below the center of the $x$-axis, the label of the
    $y$-axis is rotated by $90^\circ$ and placed left of the center of the
    $y$-axis.
    %
\begin{codeexample}[
    width=8cm,
    preamble={\usetikzlibrary{datavisualization.formats.functions}},
]
\tikz \datavisualization
 [scientific axes={clean, standard labels},
  visualize as smooth line,
  x axis={label=degree $d$,
    ticks={tick unit={}^\circ}},
  y axis={label=$\sin d$}]
data [format=function] {
  var x : interval [-10:10] samples 10;
  func y = sin(\value x);
};
\end{codeexample}
    %
\end{key}

\begin{key}{/tikz/data visualization/scientific axes/upright labels}
    Works like |scientific axes standard labels|, only the label of the
    $y$-axis is not rotated.
    %
\begin{codeexample}[
    width=8cm,
    preamble={\usetikzlibrary{datavisualization.formats.functions}},
]
\tikz \datavisualization [
  scientific axes={clean, upright labels},
  visualize as smooth line,
  x axis={label=degree $d$,
    ticks={tick unit={}^\circ}},
  y axis={label=$\cos d$, include value=1,
    ticks={style={
        /pgf/number format/precision=4,
        /pgf/number format/fixed zerofill}}}]
data [format=function] {
  var x : interval [-10:10] samples 10;
  func y = cos(\value x);
};
\end{codeexample}
    %
\end{key}

\begin{key}{/tikz/data visualization/scientific axes/end labels}
    Places the labels at the end of the $x$- and the $y$-axis, similar to the
    axis labels of a school book axis system.
    %
\begin{codeexample}[
    width=8cm,
    preamble={\usetikzlibrary{datavisualization.formats.functions}},
]
\tikz \datavisualization [
  scientific axes={clean, end labels},
  visualize as smooth line,
  x axis={label=degree $d$,
    ticks={tick unit={}^\circ}},
  y axis={label=$\tan d$}]
data [format=function] {
  var x : interval [-80:80];
  func y = tan(\value x);
};
\end{codeexample}
    %
\end{key}


\subsubsection{Reference: School Book Axis Systems}

\begin{key}{/tikz/data visualization/school book axes=\meta{options}}
    This axis system is intended to ``look like'' the coordinate systems often
    used in school books: The axes are drawn in such a way that they intersect
    to origin. Furthermore, no automatic scaling is done to ensure that the
    lengths of units are the same in all directions.

    This axis system must be used with care -- it is nearly always necessary to
    specify the desired unit length by hand using the option |unit length|. If
    the magnitudes of the units on the two axes differ, different unit lengths
    typically need to be specified for the different axes.

    Finally, if the data is ``far removed'' from the origin, this axis system
    will also ``look bad''.
    %
\begin{codeexample}[preamble={\usetikzlibrary{datavisualization.formats.functions}}]
\begin{tikzpicture}
  \datavisualization [school book axes, visualize as smooth line]
    data [format=function] {
      var x : interval [-1.3:1.3];
      func y = \value x*\value x*\value x;
    };
\end{tikzpicture}
\end{codeexample}

    The stepping of the ticks is one unit by default. Using keys like
    |ticks=some| may help to give better steppings.

    The \meta{options} are executed with the key itself as path prefix. Thus,
    the following subkeys are permissible options:
    %
    \begin{key}{/tikz/data visualization/school book axes/unit=\meta{value}}
        Sets the scaling so that 1\,cm corresponds to \meta{value} units. At
        the same time, the stepping of the ticks will also be set to
        \meta{value}.
        %
\begin{codeexample}[preamble={\usetikzlibrary{datavisualization.formats.functions}}]
\begin{tikzpicture}
  \datavisualization [school book axes={unit=10},
                      visualize as smooth line,
                      clean ticks,
                      x axis={label=$x$},
                      y axis={label=$f(x)$}]
    data [format=function] {
      var x : interval [-20:20];
      func y = \value x*\value x/10;
    };
\end{tikzpicture}
\end{codeexample}
    \end{key}

    \begin{key}{/tikz/data visualization/school book axes/standard labels}
        This key makes the label of the $x$-axis appear at the right end of
        this axis and it makes the label of the $y$-axis appear at the top of
        the $y$-axis.

        Currently, this is the only supported placement strategy for the school
        book axis system.
        %
\begin{codeexample}[preamble={\usetikzlibrary{datavisualization.formats.functions}}]
\begin{tikzpicture}
  \datavisualization [school book axes={standard labels},
                      visualize as smooth line,
                      clean ticks,
                      x axis={label=$x$},
                      y axis={label=$f(x)$}]
    data [format=function] {
      var x : interval [-1:1];
      func y = \value x*\value x + 1;
    };
\end{tikzpicture}
\end{codeexample}
    \end{key}
\end{key}


\subsubsection{Advanced Reference: Underlying Cartesian Axis Systems}

The axis systems described in the following are typically not used directly by
the user. The systems setup \emph{directions} for several axes in some sensible
way, but they do not actually draw anything on these axes. For instance, the
|xy Cartesian| creates two axes called |x axis| and |y axis| and makes the
$x$-axis point right and the $y$-axis point up. In contrast, an axis system
like |scientific axes| uses the axis system |xy Cartesian| internally and then
proceeds to setup a lot of keys so that the axis lines are drawn, ticks and
grid lines are drawn, and labels are placed at the correct positions.

\begin{key}{/tikz/data visualization/xy Cartesian}
    This axis system creates two axes called |x axis| and |y axis| that point
    right and up, respectively. By default, one unit is mapped to one cm.
    %
\begin{codeexample}[preamble={\usetikzlibrary{datavisualization.formats.functions}}]
\begin{tikzpicture}
  \datavisualization [xy Cartesian, visualize as smooth line]
    data [format=function] {
      var x : interval [-1.25:1.25];
      func y = \value x*\value x*\value x;
    };
\end{tikzpicture}
\end{codeexample}

    \begin{key}{/tikz/data visualization/xy axes=\meta{options}}
        This key applies the \meta{options} both to the |x axis| and the
        |y axis|.
    \end{key}
\end{key}

\begin{key}{/tikz/data visualization/xyz Cartesian cabinet}
    This axis system works like |xy Cartesian|, only it \emph{additionally}
    creates an axis called |z axis| that points left and down. For this axis,
    one unit corresponds to $\frac{1}{2}\sin 45^\circ\mathrm{cm}$. This is also
    known as a cabinet projection.

    \begin{key}{/tikz/data visualization/xyz axes=\meta{options}}
        This key applies the \meta{options} both to the |x axis| and the
        |y axis|.
    \end{key}
\end{key}

\begin{key}{/tikz/data visualization/uv Cartesian}
    This axis system works like |xy Cartesian|, but it introduces two axes
    called |u axis| and |v axis| rather than the |x axis| and the |y axis|. The
    idea is that in addition to a ``major'' $xy$-coordinate system this is also
    a ``smaller'' or ``minor'' coordinate system in use for depicting, say,
    small vectors with respect to this second coordinate system.

    \begin{key}{/tikz/data visualization/uv axes=\meta{options}}
        Applies the \meta{options} to both the |u axis| and the |y axis|.
    \end{key}
\end{key}

\begin{key}{/tikz/data visualization/uvw Cartesian cabinet}
    Like |xyz Cartesian cabinet|, but for the $uvw$-system.

    \begin{key}{/tikz/data visualization/uvw axes=\meta{options}}
        Like |xyz axes|.
    \end{key}
\end{key}


\subsection{Ticks and Grids}
\label{section-dv-ticks-and-grids}

\subsubsection{Concepts}

A \emph{tick} is a small visual indication on an axis of the value of the
axis's attribute at the position where the tick is shown. A tick may be
accompanied additionally by a textual representation, but it need not. A
\emph{grid line} is similar to a tick, but it is not an indication on the axis,
but rather a whole line that indicates all positions where the attribute has a
certain value. Unlike ticks, grid lines (currently) are not accompanied by a
textual representation.

Just as for axes, the data visualization system decouples the specification of
which ticks are present \emph{in principle} from where they are visualized. In
the following, I describe how you specify which ticks and grid lines you would
like to be drawn and how they should look like (their styling). The axis system
of your choice will then visualize the ticks at a sensible position for the
chosen system. For details on how to change where whole axis is shown along
with its ticks, see Section~\ref{section-dv-visualize-ticks}.

Specifying which ticks you are interested in is done as follows: First, you use
|ticks| key (or, for specifying which grid lines should be present, the |grid|
key). This key takes several possible options, described in detail in the
following, which have different effects:
%
\begin{enumerate}
    \item Keys like |step=10| or |minor steps between steps| cause a
        ``semi-automatic'' computation of possible steps. Here, you explicitly
        specify the stepping of steps, but the first stepping and their number
        are computed automatically according to the range of possible values
        for the attribute.
    \item Keys like |few|, |some|, or |many| can be passed to |ticks| in order
        to have \tikzname\ compute good tick positions automatically. This is
        usually what you want to happen, which is why most axis system will
        implicitly say |ticks={some}|.
    \item Keys like |at| or |also at| provide ``absolute control'' over which
        ticks or grid lines are shown. For these keys, you can not only specify
        at what value a tick should be shown, but also its styling and also
        whether it is a major, minor, or subminor tick or grid line.
\end{enumerate}

In the following, the main keys |ticks| and |grids| are documented first. Then
the different kinds of ways of specifying where ticks or grid lines should be
shown are explained.


\subsubsection{The Main Options: Tick and Grid}

\begin{key}{/tikz/data visualization/axis options/ticks=\meta{options} (default some)}
    This key can be passed to an axis in order to configure which ticks are
    present for the axis. The possible \meta{options} include, for instance,
    keys like |step|, which is used to specify a stepping for the ticks, but
    also keys like |major| or |minor| for specifying the positions of major and
    minor ticks in detail. The list of possible options is described in the
    rest of this section.

    Note that the |ticks| option will only configure which ticks should be
    shown in principle. The actual rendering is done only when the
    |visualize ticks| key is used, documented in
    Section~\ref{section-dv-visualize-ticks}, which is typically done only
    internally by an axis system.

    The \meta{options} will be executed with the path prefix
    |/tikz/data visualization/|. When the |ticks| key is used multiple times
    for an axis, the \meta{options} accumulate.
    %
\begin{codeexample}[width=6cm,preamble={\usetikzlibrary{datavisualization}}]
\tikz \datavisualization [
  scientific axes, visualize as line,
  x axis={ticks={step=24, minor steps between steps=3},
          label=hours}]
  data {
    x, y
    0, 0
    10, 0
    20, 0.5
    30, 0.75
    40, 0.7
    50, 0.6
    60, 0.5
    70, 0.45
    80, 0.47
  };
\end{codeexample}
    %
\end{key}

\begin{key}{/tikz/data visualization/axis options/grid=\meta{options} (default at default ticks)}
    This key is similar to |ticks|, only it is used to configure where grid
    lines should be shown rather than ticks. In particular, the options that
    can be passed to the |ticks| key can also be passed to the |grid| key. Just
    like |ticks|, the \meta{options} only specify which grid lines should be
    drawn in principle; it is the job of the |visualize grid| key to actually
    cause any grid lines to be shown.

    If you do not specify any \meta{options}, the default text
    |at default ticks| is used. This option causes grid lines to be drawn at
    all positions where ticks are shown by default. Since this usually exactly
    what you would like to happen, most of the time you just need to
    |all axes=grid| to cause a grid to be shown.
\end{key}

\begin{key}{/tikz/data visualization/axis options/ticks and grid=\meta{options}}
    This key passes the \meta{options} to both the |ticks| key and also to the
    |grid| key. This is useful when you want to specify some special points
    explicitly where you wish a tick to be shown and also a grid line.
    %
\begin{codeexample}[preamble={\usetikzlibrary{datavisualization.formats.functions}}]
\tikz \datavisualization
  [scientific axes,
   visualize as smooth line,
   all axes= {grid, unit length=1.25cm},
   y axis={ ticks=few },
   x axis={ ticks=many, ticks and grid={ major also at={(pi/2) as $\frac{\pi}{2}$}}}]
  data [format=function] {
    var x : interval [-pi/2:3*pi] samples 50;
    func y = sin(\value x r);
  };
\end{codeexample}
    %
\end{key}


\subsubsection{Semi-Automatic Computation of Tick and Grid Line Positions}
\label{section-dv-concept-tick-placement-strategies}

Consider the following problem: The data visualization engine determines that
in a plot the $x$-values vary between $17.4$ and $34.5$. In this case, we
certainly do not want, say, ten ticks at exactly ten evenly spaced positions
starting with $17.4$ and ending with $34.5$, because this would yield ticks at
positions like $32.6$. Ticks should be placed at ``nice'' positions like $20$,
$25$, and $30$.

Determining which positions are ``nice'' is somewhat difficult. In the above
example, the positions $20$, $25$, and $30$ are certainly nice, but only three
ticks may be a bit few of them. Better might be the tick positions $17.5$,
$20$, $22.5$, through to $32.5$. However, users might prefer even numbers over
fractions like $2.5$ as the stepping.

A \emph{tick placement strategy} is a method of automatically deciding which
positions are \emph{good} for placing ticks. The data visualization engine
comes with a number of predefined strategies, but you can also define new ones
yourself. When the data visualization is requested to automatically determine
``good'' positions for the placement of ticks on an axis, it uses one of
several possible \emph{basic strategies}. These strategies differ dramatically
in which tick positions they will choose: For a range of values between $5$ and
$1000$, a |linear steps| strategy might place ticks at positions $100$, $200$,
through to $1000$, while an |exponential steps| strategy would prefer the tick
positions $10$, $100$ and $1000$. The exact number and values of the tick
positions chosen by either strategy can be fine-tuned using additional options
like |step| or |about|.

Here is an example of the different stepping chosen when one varies the tick
placement strategy:
%
\begin{codeexample}[preamble={\usetikzlibrary{datavisualization.formats.functions}}]
\begin{tikzpicture}
  \datavisualization [scientific axes, visualize as smooth line]
    data [format=function] {
      var x : interval [1:11];
      func y = \value x*\value x;
    };
\end{tikzpicture}
\qquad
\begin{tikzpicture}
  \datavisualization [scientific axes, visualize as smooth line,
    y axis={exponential steps},
    x axis={ticks={quarter about strategy}},
  ]
    data [format=function] {
      var x : interval [1:11];
      func y = \value x*\value x;
    };
\end{tikzpicture}
\end{codeexample}

Two strategies are always available: |linear steps|, which yields
(semi)automatic ticks are evenly spaced positions, and |exponential steps|,
which yields (semi)automatic steps at positions at exponentially increasing
positions -- which is exactly what is needed for logarithmic plots. These
strategies are details in Section~\ref{section-dv-strategies}.

The following options are used to configure tick placement strategies like
|linear steps|. Unlike the basic choice of a placement strategy, which is an
axis option, the following should be passed to the option |ticks| or |grid|
only. So, you would write things like |x axis={ticks={step=2}}|, but
|x axis={linear steps}|.

\begin{key}{/tikz/data visualization/step=\meta{value} (initially 1)}
    The value of this key is used to determine the spacing of the major ticks.
    The key is used by the |linear steps| and |exponential steps| strategies,
    see the explanations in Section~\ref{section-dv-strategies} for details.
    Basically, all ticks are placed at all multiples of \meta{value} that lie
    in the attribute range interval.
    %
\begin{codeexample}[preamble={\usetikzlibrary{datavisualization.formats.functions}}]
\tikz \datavisualization [
    school book axes, visualize as smooth line,
    y axis={ticks={step=1.25}},
  ]
    data [format=function] {
      var x : interval [0:3];
      func y = \value x*\value x/2;
    };
\end{codeexample}
    %
\end{key}

\begin{key}{/tikz/data visualization/minor steps between steps=\meta{number} (default 9)}
    Specifies that between any two major steps (whose positions are specified
    by the |step| key), there should be \meta{number} many minor steps. Note
    that the default of |9| is exactly the right number so that each interval
    between two minor steps is exactly a tenth of the size of a major step. See
    also Section~\ref{section-dv-strategies} for further details.
    %
\begin{codeexample}[preamble={\usetikzlibrary{datavisualization.formats.functions}}]
\begin{tikzpicture}
  \datavisualization [school book axes, visualize as smooth line,
    x axis={ticks={minor steps between steps=3}},
    y axis={ticks={minor steps between steps}},
  ]
    data [format=function] {
      var x : interval [-1.5:1.5];
      func y = \value x*\value x;
    };
\end{tikzpicture}
\end{codeexample}
    %
\end{key}

\begin{key}{/tikz/data visualization/phase=\meta{value} (initially 0)}
    See Section~\ref{section-dv-strategies} for details on how the phase of
    steps influences the tick placement.
\end{key}


\subsubsection{Automatic Computation of Tick and Grid Line Positions}

The |step| option gives you ``total control'' over the stepping of ticks on an
axis, but you often do not know the correct stepping in advance. In this case,
you may prefer to have a good value for |step| being computed for you
automatically.

Like the |step| key, these options are passed to the |ticks| option. So, for
instance, you would write |x axis={ticks={about=4}}| to request about four
ticks to be placed on the $x$-axis.

\begin{key}{/tikz/data visualization/about=\meta{number}}
    This key asks the data visualization to place \emph{about} \meta{number}
    many ticks on an axis. It is not guaranteed that \emph{exactly}
    \meta{number} many ticks will be used, rather the actual number will be the
    closest number of ticks to \meta{number} so that their stepping is still
    ``good''. For instance, when you say |about=10|, it may happen that exactly
    |10|, but perhaps even |13| ticks are actually selected, provided that
    these numbers of ticks lead to good stepping values like |5| or |2.5|
    rather than numbers like |3.4| or |7|. The method that is used to determine
    which steppings a deemed to be ``good'' depends on the current tick
    placement strategy.


    \medskip
    \textbf{Linear steps.}
    Let us start with |linear steps|: First, the difference between the maximum
    value $v_{\max}$ and the minimum value $v_{\min}$ on the axis is computed;
    let us call it $r$ for ``range''. Then, $r$ is divided by \meta{number},
    yielding a target stepping~$s$. If $s$ is a number like $1$ or $5$ or $10$,
    then this number could be used directly as the new value of |step|.
    However, $s$ will typically something strange like $0.023\,45$ or
    $345\,223.76$, so $s$ must be replaced by a better value like $0.02$ in the
    first case and perhaps $250\,000$ in the second case.

    In order to determine which number is to be used, $s$ is rewritten in the
    form $m \cdot 10^k$ with $1 \le m < 10$ and $k \in \mathbb Z$. For
    instance, $0.023\,45$ would be rewritten as $2.345 \cdot 10^{-2}$ and
    $345\,223.76$ as $3.452\,2376 \cdot 10^5$. The next step is to replace the
    still not-so-good number $m$ like $2.345$ or $3.452\,237$ by a ``good''
    value $m'$. For this, the current value of the |about strategy| is used:
    %
    \begin{key}{/tikz/data visualization/about strategy=\meta{list}}
        The \meta{list} is a comma-separated sequence of pairs
        \meta{threshold}/\meta{value} like for instance |1.5/1.0| or |2.3/2.0|.
        When a good value $m'$ is sought for a given $m$, we iterate over the
        list and find the first pair \meta{threshold}/\meta{value} where
        \meta{threshold} exceeds~$m$. Then $m'$ is set to \meta{value}. For
        instance, if \meta{list} is |1.5/1.0,2.3/2.0,4/2.5,7/5,11/10|, which is
        the default, then for $m=3.141$ we would get $m'=2.5$ since $4 >
        3.141$, but $2.3 \le 3.141$. For $m=6.3$ we would get $m'=5$.
    \end{key}
    %
    Once $m'$ has been determined, the stepping is set to $s' = m' \cdot 10^k$.

    % Define an axis type
    \tikzdatavisualizationset{
      one dimensional axis/.style={
        new Cartesian axis=axis,
        axis={
          attribute=main,
          unit vector={(0pt,1pt)},
          visualize axis={style=->},
          visualize ticks={major={tick text at low},direction axis=perpendicular},
          length=3cm
        },
        new Cartesian axis=perpendicular,
        perpendicular={
          attribute=perp,
          unit vector={(1pt,0pt)},
          include value=0,
          include value=1
        }
      }
    }

    \def\showstrategy#1{
        % Show the effect for the different strategies
        \medskip
        \begin{tikzpicture}
          \foreach \max/\about [count=\c] in {10/5,20/5,30/5,40/5,50/5,60/5,70/5,80/5,90/5,100/5,100/3,100/10}
          {
            \begin{scope}[xshift=\c pt*30]
              \datavisualization [#1,
              one dimensional axis,
              axis={
                ticks={about=\about},
                include value=0,
                include value=\max
              }
              ];

              \node at (0,-5mm) [anchor=mid] {\texttt{\about}};
            \end{scope}
          }

          \node at (30pt,-5mm) [anchor=mid east] {\texttt{about=\ \ }};
      \end{tikzpicture}
    }

    The net effect of all this is that for the default strategy the only valid
    stepping are the values $1$, $2$, $2.5$ and $5$ and every value obtainable
    by multiplying one of these values by a power of ten. The following example
    shows the effects of, first, setting |about=5| (corresponding to the |some|
    option) and then having axes where the minimum value is always |0| and
    where the maximum value ranges from |10| to |100| and, second, setting
    |about| to the values from |3| (corresponding to the |few| option) and to
    |10| (corresponding to the |many| option) while having the minimum at |0|
    and the maximum at |100|:

    \showstrategy{standard about strategy}

    \medskip
    \textbf{Exponential steps.}
    For |exponential steps| the strategy for determining a good stepping value
    is similar to |linear steps|, but with the following differences:
    %
    \begin{itemize}
        \item Naturally, since the stepping value refers to the exponent, the
            whole computation of a good stepping value needs to be done ``in
            the exponent''. Mathematically spoken, instead of considering the
            difference $r = v_{\max} - v_{\min}$, we consider the difference $r
            = \log v_{\max} - \log v_{\min}$. With this difference, we still
            compute $s = r / \meta{number}$ and let $s = m \cdot 10^k$ with $1
            \le m < 10$.
        \item It makes no longer sense to use values like $2.5$ for $m'$ since
            this would yield a fractional exponent. Indeed, the only sensible
            values for $m'$ seem to be $1$, $3$, $6$, and $10$. Because of
            this, the |about strategy| is ignored and one of these values or a
            multiple of one of them by a power of ten is used.
    \end{itemize}

    The following example shows the chosen steppings for a maximum varying from
    $10^1$ to $10^5$ and from $10^{10}$ to $10^{50}$ as well as for $10^{100}$
    for |about=3|:

    \medskip
    \begin{tikzpicture}
      \foreach \max [count=\c] in {1,...,5,10,20,...,50,100}
        {
          \begin{scope}[xshift=\c pt*40]
            \datavisualization [
            one dimensional axis,
            axis={
              logarithmic,
              ticks={about=3},
              include value=1,
              include value=1e\max
            }
            ];
          \end{scope}
        }
    \end{tikzpicture}


    \medskip
    \textbf{Alternative strategies.}

    In addition to the standard |about strategy|, there are some additional
    strategies that you might wish to use instead:

    \begin{key}{/tikz/data visualization/standard about strategy}
        Permissible values for $m'$ are: $1$, $2$, $2.5$, and~$5$. This
        strategy is the default strategy.
    \end{key}

    \begin{key}{/tikz/data visualization/euro about strategy}
        Permissible values for $m'$ are: $1$, $2$, and~$5$. These are the same
        values as for the Euro coins, hence the name.

        \showstrategy{euro about strategy}
    \end{key}

    \begin{key}{/tikz/data visualization/half about strategy}
        Permissible values for $m'$: $1$ and $5$. Use this strategy if only
        powers of $10$ or halves thereof seem logical.

        \showstrategy{half about strategy}
    \end{key}

    \begin{key}{/tikz/data visualization/decimal about strategy}
        The only permissible value for $m'$ is $1$. This is an even more
        radical version of the previous strategy.

        \showstrategy{decimal about strategy}
    \end{key}

    \begin{key}{/tikz/data visualization/quarter about strategy}
        Permissible values for $m'$ are: $1$, $2.5$, and $5$.

        \showstrategy{quarter about strategy}
    \end{key}

    \begin{key}{/tikz/data visualization/int about strategy}
        Permissible values for $m'$ are: $1$, $2$, $3$, $4$, and $5$.

        \showstrategy{int about strategy}
    \end{key}
\end{key}

\begin{key}{/tikz/data visualization/many}
    This is an abbreviation for |about=10|.
\end{key}

\begin{key}{/tikz/data visualization/some}
    This is an abbreviation for |about=5|.
\end{key}

\begin{key}{/tikz/data visualization/few}
    This is an abbreviation for |about=3|.
\end{key}

\begin{key}{/tikz/data visualization/none}
    Switches off the automatic step computation. Unless you use |step=|
    explicitly to set a stepping, no ticks will be (automatically) added.
\end{key}


\subsubsection{Manual Specification of Tick and Grid Line Positions}

The automatic computation of ticks and grid lines will usually do a good job,
but not always. For instance, you might wish to have ticks exactly at, say,
prime numbers or at Fibonacci numbers or you might wish to have an additional
tick at $\pi$. In these cases you need more direct control over the
specification of tick positions.

First, it is important to understand that the data visualization system
differentiates between three kinds of ticks and grid lines: major, minor, and
subminor. The major ticks are the most prominent ticks where, typically, a
textual representation of the tick is shown; and the major grid lines are the
thickest. The minor ticks are smaller, more numerous, and lie between major
ticks. They are used, for instance, to indicate positions in the middle between
major ticks or at all integer positions between major ticks. Finally, subminor
ticks are even smaller than minor ticks and they lie between minor ticks.

Four keys are used to configure the different kinds:

\begin{key}{/tikz/data visualization/major=\meta{options}}
    The key can be passed as an option to the |ticks| key and also to the
    |grid| key, which in turn is passed as an option to an axis. The
    \meta{options} passed to |major| specify at which positions major
    ticks/grid lines should be shown (using the |at| option and |also at|
    option) and also any special styling. The different possible options are
    described later in this section.
    %
\begin{codeexample}[preamble={\usetikzlibrary{datavisualization.formats.functions}}]
\tikz \datavisualization
  [ school book axes, visualize as smooth line,
    x axis={ticks={major={at={1, 1.5, 2}}}}]
  data [format=function] {
    var x : interval [-1.25:2];
    func y = \value x * \value x / 2;
  };
\end{codeexample}
    %
\end{key}

\begin{key}{/tikz/data visualization/minor=\meta{options}}
    Like |major|, only for minor ticks/grid lines.
    %
\begin{codeexample}[preamble={\usetikzlibrary{datavisualization.formats.functions}}]
\tikz \datavisualization
  [ school book axes, visualize as smooth line,
    x axis={grid={minor={at={1, 1.5, 2}}}}]
  data [format=function] {
    var x : interval [-1.25:2];
    func y = \value x * \value x / 2;
  };
\end{codeexample}
    %
\end{key}

\begin{key}{/tikz/data visualization/subminor=\meta{options}}
    Like |major|, only for subminor ticks/grid lines.
\end{key}

\begin{key}{/tikz/data visualization/common=\meta{options}}
    This key allows you to specify \meta{options} that apply to |major|,
    |minor| and |subminor| alike. It does not make sense to use |common| to
    specify positions (since you typically do not want both a major and a minor
    tick at the same position), but it can be useful to configure, say, the
    size of all kinds of ticks:
    %
\begin{codeexample}[preamble={\usetikzlibrary{datavisualization.formats.functions}}]
\tikz \datavisualization
  [ school book axes, visualize as smooth line,
    x axis={ticks={minor steps between steps, common={low=0}}} ]
  data [format=function] {
    var x : interval [-1.25:2];
    func y = \value x * \value x / 2;
  };
\end{codeexample}
    %
\end{key}

The following keys can now be passed to the |major|, |minor|, and |subminor|
keys to specify where ticks or grid lines should be shown:

\begin{key}{/tikz/data visualization/at=\meta{list}}
    Basically, the \meta{list} must be a list of values that is processed with
    the |\foreach| macro (thus, it can contain ellipses to specify ranges of
    value). Empty values are skipped.

    The effect of passing |at| to a |major|, |minor|, or |subminor| key is that
    ticks or grid lines on the axis will be placed exactly at the values in
    \meta{list}. Here is an example:
    %
\begin{codeexample}[preamble={\usetikzlibrary{datavisualization.formats.functions}}]
\tikz \datavisualization
  [ school book axes, visualize as smooth line,
    x axis={ticks={major={at={-1,0.5,(pi/2)}}}}]
  data [format=function] {
    var x : interval [-1.25:2];
    func y = \value x * \value x / 2;
  };
\end{codeexample}
    When this option is used, any previously specified tick positions are
    overwritten by the values in \meta{list}. Automatically computed ticks are
    also overwritten. Thus, this option gives you complete control over where
    ticks should be placed.

    Normally, the individual values inside the \meta{list} are just numbers
    that are specified in the same way as an attribute value. However, such a
    value may also contain the keyword |as|, which allows you so specify the
    styling of the tick in detail. Section~\ref{section-dv-ticks-styling}
    details how this works.

    It is often a bit cumbersome that one has to write things like
    %
\begin{codeexample}[code only]
some axis = {ticks = {major = {at = {...}}}}
\end{codeexample}
    %
    A slight simplification is given by the following keys, which can be passed
    directly to |ticks| and |grid|:
    %
    \begin{key}{/tikz/data visualization/major at=\meta{list}}
        A shorthand for |major={at={|\meta{list}|}}|.
    \end{key}
    %
    \begin{key}{/tikz/data visualization/minor at=\meta{list}}
        A shorthand for |major={at={|\meta{list}|}}|.
    \end{key}
    %
    \begin{key}{/tikz/data visualization/subminor at=\meta{list}}
        A shorthand for |major={at={|\meta{list}|}}|.
    \end{key}
\end{key}

\begin{key}{/tikz/data visualization/also at=\meta{list}}
    This key is similar to |at|, but it causes ticks or grid lines to be placed
    at the positions in the \meta{list} \emph{in addition} to the ticks that
    have already been specified either directly using |at| or indirectly using
    keys like |step| or |some|. The effect of multiple calls of this key
    accumulate. However, when |at| is used after an |also at| key, the |at| key
    completely resets the positions where ticks or grid lines are shown.
    %
\begin{codeexample}[preamble={\usetikzlibrary{datavisualization.formats.functions}}]
\tikz \datavisualization
  [ school book axes, visualize as smooth line,
    x axis={grid, ticks and grid={major={also at={0.5}}}}]
  data [format=function] {
    var x : interval [-1.25:2];
    func y = \value x * \value x / 2;
  };
\end{codeexample}
    %
    As for |at|, there are some shorthands available:
    %
    \begin{key}{/tikz/data visualization/major also at=\meta{list}}
        A shorthand for |major={also at={|\meta{list}|}}|.
    \end{key}
    %
    \begin{key}{/tikz/data visualization/minor also at=\meta{list}}
        A shorthand for |major={also at={|\meta{list}|}}|.
    \end{key}
    %
    \begin{key}{/tikz/data visualization/subminor also at=\meta{list}}
        A shorthand for |major={also at={|\meta{list}|}}|.
    \end{key}
\end{key}


\subsubsection{Styling Ticks and Grid Lines: Introduction}
\label{section-dv-ticks-styling}

When a tick, a tick label, or a grid line is visualized on the page, a whole
regiment of styles influences the appearance. The reason for this large number
of interdependent styles is the fact that we often wish to influence only a
very certain part of how a tick is rendered while leaving the other aspects
untouched: Sometimes we need to modify just the font of the tick label;
sometimes we wish to change the length of the tick label and the tick label
position at the same time; sometimes we wish to change the color of grid line,
tick, and tick label; and sometimes we wish to generally change the thickness
of all ticks.

Let us go over the different kinds of things that can be styled (grid lines,
ticks, and tick labels) one by one and let us have a look at which styles are
involved. We will start with the grid lines, since they turn out to be the most
simple, but first let us have a look at the general |style| and |styling|
mechanism that is used in many placed in the following:


\subsubsection{Styling Ticks and Grid Lines: The Style and Node Style Keys}
\label{section-dv-style}

All keys of the data visualization system have the path prefix
|/tikz/data visualization|. This is not only true for the main keys like
|scientific axes| or |visualize as line|, but also for keys that govern how
ticks are visualized. In particular, a style like |every major grid| has the
path prefix |/tikz/data visualization| and all keys stored in this style are
also executed with this path prefix.

Normally, this does not cause any trouble since most of the keys and even
styles used in a data visualization are intended to configure what is shown in
the visualization. However, at some point, we may also with to specify options
that no longer configure the visualization in general, but specify the
appearance of a line or a node on the \tikzname\ layer.

Two keys are used to ``communicate'' with the \tikzname\ layer:

\begin{key}{/tikz/data visualization/style=\meta{\tikzname\ options}}
    This key takes options whose path prefix is |/tikz|, not
    |/tikz/data visualization|. These options will be \emph{appended} to a
    current list of such options (thus, multiple calls of this key accumulate).
    The resulting list of keys is not executed immediately, but it will be
    executed whenever the data visualization engine calls the \tikzname\ layer
    to draw something (this placed will be indicated in the following).
    %
\begin{codeexample}[preamble={\usetikzlibrary{datavisualization.formats.functions}}]
\tikz \datavisualization
  [scientific axes,
   all axes={ticks={style=blue}, length=3cm},
   y axis={grid, grid={minor steps between steps, major={style=red}}},
   visualize as line]
  data [format=function] {
    var x : interval [5:10];
    func y = \value x * \value x;
  };
\end{codeexample}
    %
\end{key}

\begin{key}{/tikz/data visualization/styling}
    Executing this key will cause all ``accumulated'' \tikzname\ options from
    previous calls to the key |/tikz/data visualization/style| to be executed.
    Thus, you use |style| to set \tikzname\ options, but you use |styling| to
    actually apply these options. Usually, you do not call this option directly
    since this application is only done deep inside the data visualization
    engine.
\end{key}

Similar to |style| (and |styling|) there also exist the |node style| (and
|node styling|) key that takes \tikzname\ options that apply to nodes only --
in addition to the usual |style|.

\begin{key}{/tikz/data visualization/node style=\meta{\tikzname\ options}}
    This key works like |style|, but it has an effect only on nodes that are
    created during a data visualization. This includes tick labels and axis
    labels:
    %
\begin{codeexample}[preamble={\usetikzlibrary{datavisualization.formats.functions}}]
\tikz \datavisualization
  [scientific axes,
   all axes={ticks={node style=red}, length=3cm},
   visualize as line]
  data [format=function] {
    var x : interval [5:10];
    func y = \value x * \value x;
  };
\end{codeexample}
    %
    Note that in the example the ticks themselves (the little thicker lines)
    are not red.
\end{key}

\begin{key}{/tikz/data visualization/node styling}
    Executing this key will cause all ``accumulated'' node stylings to be
    executed.
\end{key}


\subsubsection{Styling Ticks and Grid Lines: Styling Grid Lines}
\label{section-dv-styling-grid-lines}

When a grid line is visualized, see
Section~\ref{section-dv-visualize-gridlines} for details on when this happens,
the following styles are executed in the specified order.
%
\begin{enumerate}
    \item |grid layer|.
    \item |every grid|.
    \item |every major grid| or |every minor grid| or |every subminor grid|,
        depending on the kind of grid line.
    \item locally specified options for the individual grid line, see
        Section~\ref{section-dv-local-styles}.
    \item |styling|, see Section~\ref{section-dv-style}.
\end{enumerate}

All of these keys have the path prefix |/tikz/data visualization|. However, the
options stored in the first style (|grid layer|) and also in the last
(|styling|) are executed with the path prefix |/tikz| (see
Section~\ref{section-dv-style}).

Let us now have a look at these keys in detail:

\begin{stylekey}{/tikz/data visualization/grid layer (initially on background layer)}
\label{section-dv-grid-layer}%
    This key is used to specified the \emph{layer} on which grid lines should
    be drawn (layers are explained in Section~\ref{section-tikz-backgrounds}).
    By default, all grid lines are placed on the |background| layer and thus
    behind the data visualization. This is a sensible strategy since it avoids
    obscuring the more important data with the far less important grid lines.
    However, you can change this style to ``get the grid lines to the front'':
    %
\begin{codeexample}[preamble={\usetikzlibrary{datavisualization.formats.functions}}]
\tikz \datavisualization
  [scientific axes,
   all axes={
     length=3cm,
     grid,
     grid={minor steps between steps}
   },
   grid layer/.style=, % none, so on top of data (bad idea)
   visualize as line]
  data [format=function] {
    var x : interval [5:10];
    func y = \value x * \value x;
  };
\end{codeexample}
    %
    When this style is executed, the keys stored in the style will be executed
    with the prefix |/tikz|. Normally, you should only set this style to be
    empty or to |on background layer|.
\end{stylekey}

\begin{stylekey}{/tikz/data visualization/every grid}
    This style provides overall configuration options for grid lines. By
    default, it is set to the following:
    %
\begin{codeexample}[code only]
low=min, high=max
\end{codeexample}
    %
    This causes grid lines to span all possible values when they are
    visualized, which is usually the desired behavior (the |low| and |high|
    keys are explained in Section~\ref{section-dv-visualize-ticks}. You can
    append the |style| key to this style to configure the overall appearance of
    grid lines. It should be noted that settings to |style| inside |every grid|
    will take precedence over ones in |every major grid| and |every minor grid|.
    In the following example we cause all grid lines to be dashed (which is not
    a good idea in general since it creates a distracting background pattern).
    %
\begin{codeexample}[preamble={\usetikzlibrary{datavisualization.formats.functions}}]
\tikz \datavisualization
  [scientific axes,
   all axes={length=3cm, grid},
   every grid/.append style={style=densely dashed},
   visualize as line]
  data [format=function] {
    var x : interval [5:10];
    func y = \value x * \value x;
  };
\end{codeexample}
    %
\end{stylekey}

\begin{stylekey}{/tikz/data visualization/every major grid}
    This style configures the appearance of major grid lines. It does so by
    calling the |style| key to setup appropriate \tikzname\ options for
    visualizing major grid lines. The default definition of this style is:
    %
\begin{codeexample}[code only]
style = {help lines, thin, black!25}
\end{codeexample}
    %
    In the following example, we use thin major blue grid lines:
    %
\begin{codeexample}[preamble={\usetikzlibrary{datavisualization.formats.functions}}]
\tikz \datavisualization
  [scientific axes,
   all axes={
     length=3cm,
     grid,
     grid={minor steps between steps}
   },
   every major grid/.style = {style={blue, thin}},
   visualize as line]
  data [format=function] {
    var x : interval [5:10];
    func y = \value x * \value x;
  };
\end{codeexample}
    %
    As can be seen, this is not exactly visually pleasing. The default settings
    for the grid lines should work in most situations; you may wish to increase
    the blackness level, however, when you experience trouble during printing
    or projecting graphics.
\end{stylekey}

\begin{stylekey}{/tikz/data visualization/every minor grid}
    Works like |every major grid|. The default is
    %
\begin{codeexample}[code only]
style = {help lines, black!25}
\end{codeexample}
    %
\end{stylekey}

\begin{stylekey}{/tikz/data visualization/every subminor grid}
    Works like |every major grid|. The default is
    %
\begin{codeexample}[code only]
style = {help lines, black!10}
\end{codeexample}
    %
\end{stylekey}


\subsubsection{Styling Ticks and Grid Lines: Styling Ticks and Tick Labels}
\label{section-dv-styling-ticks}

Styling ticks and tick labels is somewhat similar to styling grid lines. Let us
start with the tick \emph{mark}, that is, the small line that represents the
tick. When this mark is drawn, the following styles are applied:
%
\begin{enumerate}
    \item |every ticks|.
    \item |every major ticks| or |every minor ticks| or |every subminor ticks|,
        depending on the kind of ticks to be visualized.
    \item locally specified options for the individual tick, see
        Section~\ref{section-dv-local-styles}.
    \item |tick layer|
    \item |every odd tick| or |every even tick|, see
        Section~\ref{section-dv-stacking}.
    \item |draw|
    \item |styling|, see Section~\ref{section-dv-style}.
\end{enumerate}

For the tick label node (the node containing the textual representation of the
attribute's value at the tick position), the following styles are applied:
%
\begin{enumerate}
    \item |every ticks|.
    \item |every major ticks| or |every minor ticks| or |every subminor ticks|,
        depending on the kind of ticks to be visualized.
    \item locally specified options for the individual tick, see
        Section~\ref{section-dv-local-styles}.
    \item |tick node layer|
    \item |every odd tick| or |every even tick|, see
        Section~\ref{section-dv-stacking}.
    \item |styling|, see Section~\ref{section-dv-style}.
    \item |node styling|, see Section~\ref{section-dv-style}.
\end{enumerate}

\begin{stylekey}{/tikz/data visualization/every ticks}
    This style allows you to configure the appearance of ticks using the
    |style| and |node style| key. Here is (roughly) the default definition of
    this style:
    %
\begin{codeexample}[code only]
node style={
  font=\footnotesize,
  inner sep=1pt,
  outer sep=.1666em,
  rounded corners=1.5pt
}
\end{codeexample}
    %
\end{stylekey}

\begin{stylekey}{/tikz/data visualization/every major ticks}
    The default is
    %
\begin{codeexample}[code only]
  style={line cap=round}, tick length=2pt
\end{codeexample}
    %
\end{stylekey}

\begin{stylekey}{/tikz/data visualization/every minor ticks}
    The default is
    %
\begin{codeexample}[code only]
  style={help lines,thin, line cap=round}, tick length=1.4pt
\end{codeexample}
    %
\end{stylekey}

\begin{stylekey}{/tikz/data visualization/every subminor ticks}
    The default is
    %
\begin{codeexample}[code only]
  style={help lines, line cap=round}, tick length=0.8pt
\end{codeexample}
    %
\end{stylekey}

\begin{stylekey}{/tikz/data visualization/tick layer (initially on background layer)}
    Like |grid layer|, this key specifies on which layer the ticks should be
    placed.
\end{stylekey}

\begin{stylekey}{/tikz/data visualization/tick node layer (initially \normalfont empty)}
    Like |tick layer|, but now for the nodes. By default, tick nodes are placed
    on the main layer and thus on top of the data in case that the tick nodes
    are inside the data.
\end{stylekey}


\subsubsection{Styling Ticks and Grid Lines: Exceptional Ticks}

You may sometimes wish to style a few ticks differently from the other ticks.
For instance, in the axis system |school book axes| there should be a tick
label at the |0| position only on one axis and then this label should be offset
a bit. In many cases this is easy to achieve: When you add a tick ``by hand''
using the |at| or |also at| option, you can add any special options in square
brackets.

However, in some situations the special tick position has been computed
automatically for you, for instance by the |step| key or by saying |tick=some|.
In this case, adding a tick mark with the desired options using |also at| would
cause the tick mark with the correct options to be shown in addition to the
tick mark with the wrong options. In cases like this one, the following option
may be helpful:

\begin{key}{/tikz/data visualization/options at=\meta{value} |as [|\meta{options}|]|}
    This key causes the \meta{options} to be executed for any tick mark(s) at
    \meta{value} in addition to any options given already for this position:
    %
\begin{codeexample}[
    width=7cm,
    preamble={\usetikzlibrary{datavisualization.formats.functions}},
]
\tikz \datavisualization [
  scientific axes,
  visualize as smooth line,
  x axis={ticks={major={
    options at = 3    as [no tick text],
    also at    = (pi) as
      [{tick text padding=1ex}] $\pi$}}}]
data [format=function] {
  var x : interval[0:2*pi];
  func y = sin(\value x r);
};
\end{codeexample}
    %
\end{key}

\begin{key}{/tikz/data visualization/no tick text at=\meta{value}}
    Shorthand for |options at=|\meta{value}| as [no tick text]|.
\end{key}


\subsubsection{Styling Ticks and Grid Lines: Styling and Typesetting a Value}
\label{section-dv-local-styles}
\label{section-dv-tick-labels}

The \todosp{why 2 labels?} |at| and |also at| key allow you to provide a
comma-separated \meta{list} of \meta{value}s where ticks or grid lines should
be placed. In the simplest case, the \meta{value} is simply a number. However,
the general syntax allows three different kinds of \meta{value}s:
%
\begin{enumerate}
    \item \meta{value}
    \item \meta{value} |as| |[|\meta{local options}|]|
    \item \meta{value} |as| \opt{|[|\meta{local options}|]|} \meta{text}
\end{enumerate}

In the first case, the \meta{value} is just a number that is interpreted like
any other attribute value.

In the second case, where the keyword |as| is present, followed by some option
in square brackets, but nothing following the closing square bracket, when the
tick or grid line at position \meta{value} is shown, the \meta{local options}
are executed first. These can use the |style| key or the |node style| key to
configure the appearance of this single tick or grid line. You can also use
keys like |low| or |high| to influence how large the grid lines or the ticks
are or keys like |tick text at low| to explicitly hide or show a tick label.

In the third case, which is only important for |ticks| and not for |grid|, the
same happens as in the second case, but the text that is shown as tick label is
\meta{text} rather than the automatically generated tick label. This automatic
generation of tick labels is explained in the following.
%
\begin{codeexample}[preamble={\usetikzlibrary{datavisualization.formats.functions}}]
\tikz \datavisualization
  [scientific axes=clean,
   x axis={length=2.5cm, ticks={major at={
         5,
         6 as [style=red],
         7 as [{style=blue, low=-1em}],
         8 as [style=green] $2^3$,
         10 as ten
       }}},
   visualize as line]
  data [format=function] {
    var x : interval [5:10];
    func y = \value x * \value x;
  };
\end{codeexample}

A value like ``2'' or ``17'' could just be used as \meta{text} to be displayed
in the node of a tick label. However, things are more difficult when the
to-be-shown value is $0.0000000015$, because we then would typically (but not
always) prefer something like $1.5 \cdot 10^{-9}$ to be shown. Also, we might
wish a unit to be added like $23\mathrm{m}/\mathrm{s}$. Finally, we might wish
a number like $3.141$ to be replaced by $\pi$. For these reasons, the data
visualization system does not simply put the to-be-shown value in a node as
plain text. Instead, the number is passed to a \emph{typesetter} whose job it
is to typeset this number nicely using \TeX's typesetting capabilities. The
only exception is, as indicated above, the third syntax version of the |at| and
|also at| keys, where \meta{text} is placed in the tick label's node,
regardless of what the typesetting would usually do.

The text produced by the automatic typesetting is computed as follows:
%
\begin{enumerate}
    \item The current contents of the key |tick prefix| is put into the node.
    \item This is followed by a call of the key |tick typesetter| which gets
        the \meta{value} of the tick as its argument in scientific notation.
    \item This is followed by the contents of the key |tick suffix|.
\end{enumerate}

Let us have a look at these keys in detail:

\begin{key}{/tikz/data visualization/tick prefix=\meta{text} (initially \normalfont empty)}
    The \meta{text} will be put in front of every typeset tick:
    %
\begin{codeexample}[preamble={\usetikzlibrary{datavisualization.formats.functions}}]
\tikz \datavisualization
  [scientific axes, all axes={ticks=few, length=2.5cm},
   x axis={ticks={tick prefix=$\langle$, tick suffix=$]$}},
   visualize as line]
  data [format=function] {
    var x : interval [5:10];
    func y = \value x * \value x;
  };
\end{codeexample}
    %
\end{key}

\begin{key}{/tikz/data visualization/tick suffix=\meta{text} (initially \normalfont empty)}
    Works like |tick prefix|. This key is especially useful for adding units
    like ``cm'' or ``$\mathrm m/\mathrm s$'' to every tick label. For this
    reason, there is a (near) alias that is easier to memorize:
    %
    \begin{key}{/tikz/data visualization/tick unit=\meta{roman math text}}
        A shorthand for |tick suffix={$\,\rm|\meta{roman math text}|$}|:
        %
\begin{codeexample}[preamble={\usetikzlibrary{datavisualization.formats.functions}}]
\tikz \datavisualization
  [scientific axes, all axes={length=3cm},
   x axis={ticks={tick unit=s}},
   y axis={ticks={tick unit=m/s^2}},
   visualize as line]
  data [format=function] {
    var x : interval [5:10];
    func y = \value x * \value x;
  };
\end{codeexample}
    \end{key}
\end{key}

\begin{key}{/tikz/data visualization/tick typesetter=\meta{value}}
    The key gets called for each number that should be typeset. The argument
    \meta{value} will be in scientific notation (like |1.0e1| for $10$). By
    default, this key applies |\pgfmathprintnumber| to its argument. This
    command is a powerful number printer whose configuration is documented in
    Section~\ref{pgfmath-numberprinting}.

    You are invited to code underlying this key so that a different typesetting
    mechanism is used. Here is a (not quite finished) example that shows how,
    say, numbers could be printed in terms of multiples of $\pi$:
    %
\begin{codeexample}[preamble={\usetikzlibrary{datavisualization.formats.functions}}]
\def\mytypesetter#1{%
  \pgfmathparse{#1/pi}%
  \pgfmathprintnumber{\pgfmathresult}$\pi$%
}
\tikz \datavisualization
  [school book axes, all axes={unit length=1.25cm},
   x axis={ticks={step=(0.5*pi), tick typesetter/.code=\mytypesetter{##1}}},
   y axis={include value={-1,1}},
   visualize as smooth line]
  data [format=function] {
    var x : interval [0.5:7];
    func y = sin(\value x r);
  };
\end{codeexample}
    %
\end{key}


\subsubsection{Stacked Ticks}
\label{section-dv-stacking}

Sometimes, the text of tick labels are so long or so numerous that the text of
adjacent tick labels overlap (or have too little padding):
%
\begin{codeexample}[preamble={\usetikzlibrary{datavisualization.formats.functions}}]
\tikz \datavisualization [scientific axes,
                          all axes={length=2.5cm},
                          visualize as smooth line]
  data [format=function] {
    var y : interval[-100:100];
    func x = \value y*\value y;
  };
\end{codeexample}
%
There are two ways to address this problem:
%
\begin{itemize}
    \item One can rotate the labels on horizontal axes:
        %
\begin{codeexample}[preamble={\usetikzlibrary{datavisualization.formats.functions}}]
\tikz \datavisualization [scientific axes,
                          all axes={length=2.5cm},
                          x axis={ticks={node style={rotate=90, anchor=east}}},
                          visualize as smooth line]
  data [format=function] {
    var y : interval[-100:100];
    func x = \value y*\value y;
  };
\end{codeexample}
        %
        This is often a good solution, but may be hard to read. Also consider
        rotating labels only by $45^\circ$ or $30^\circ$.
    \item One can specify different shifts of the nodes for the different
        ticks, whereby the ticks text no longer overlap.
        %
\begin{codeexample}[preamble={\usetikzlibrary{datavisualization.formats.functions}}]
\tikz \datavisualization [scientific axes,
                          all axes={length=2.5cm},
                          x axis={ticks={major at={0,4000,8000,
                                2000 as [node style={yshift=-1em}],
                                6000 as [node style={yshift=-1em}],
                                10000 as [node style={yshift=-1em}]}}},
                          visualize as smooth line]
  data [format=function] {
    var y : interval[-100:100];
    func x = \value y*\value y;
  };
\end{codeexample}
        %
        However, specifying shifts ``by hand'' in the above way is not always
        an option, especially when the tick positions should be computed
        automatically. Instead, the |stack| option can be used, which is much
        easier to use and gives better results:
        %
\begin{codeexample}[preamble={\usetikzlibrary{datavisualization.formats.functions}}]
\tikz \datavisualization [scientific axes,
                          all axes={length=2.5cm}, x axis={ticks=stack},
                          visualize as smooth line]
  data [format=function] {
    var y : interval[-100:100];
    func x = \value y*\value y;
  };
\end{codeexample}
        %
\end{itemize}

The |stack| option is actually just a style that gives you access to the
general even/odd mechanism for ticks with labels. Whenever a tick mark is
created where a tick label is also to be drawn, two special things happen:
%
\begin{enumerate}
    \item For every odd tick mark, the |every odd tick| style is executed, for
        every even tick mark the |every even tick|. Here, ``odd'' and ``even''
        are with respect to the order in which the ticks have been added to the
        list of |at| positions for each major, minor, or subminor tick list,
        not with respect to the order in which they will appear on the axis.
        Thus, when you write
        %
\begin{codeexample}[code only]
ticks={major at={1,2,3,4}, major at={0,-1,-2}, minor at={9,8,7}}
\end{codeexample}
        %
        then for |1|, |3|, |0|, and |-2| as well as |9| and |7| the key
        |every odd tick| will be executed, while |every even tick| will be
        executed for positions |2|, |4|, |-1|, and also |8|.
    \item When a tick node label is shown at the |low| position of the tick
        mark, the dimension stored in the key |tick text low even padding| is
        added to the |low| value. Provided that this padding is not zero (which
        is the default), the length of the even tick marks will be increased
        and the tick label node will be placed at a greater distance from the
        axis.

        Similar keys exist for padding ticks with labels at high positions and
        also at even positions.
\end{enumerate}

\begin{key}{/tikz/data visualization/tick text low even padding=\meta{dimension} (initially 0pt)}
    When a tick label is shown at the low position of an even tick, the
    \meta{distance} is added to the |low| value, see also
    Section~\ref{section-dv-visualize-ticks}.
    %
\begin{codeexample}[preamble={\usetikzlibrary{datavisualization.formats.functions}}]
\tikz \datavisualization [scientific axes,
                          all axes={length=2.5cm},
                          x axis={ticks={tick text low even padding=-1em}},
                          visualize as smooth line]
  data [format=function] {
    var y : interval[-100:100];
    func x = \value y*\value y;
  };
\end{codeexample}
    %
    Note that \meta{dimension} should usually be non-positive.
\end{key}

The following keys work similarly:
%
\begin{key}{/tikz/data visualization/tick text low odd padding=\meta{dimension} (initially 0pt)}
\end{key}
%
\begin{key}{/tikz/data visualization/tick text high even padding=\meta{dimension} (initially 0pt)}
\end{key}
%
\begin{key}{/tikz/data visualization/tick text high odd padding=\meta{dimension} (initially 0pt)}
\end{key}

\begin{key}{/tikz/data visualization/tick text odd padding=\meta{dimension}}
    A shorthand for setting |tick text odd low padding| and
    |tick text odd high padding| at the same time.
\end{key}

\begin{key}{/tikz/data visualization/tick text even padding=\meta{dimension}}
    A shorthand for setting |tick text even low padding| and
    |tick text even high padding| at the same time.
\end{key}

\begin{key}{/tikz/data visualization/tick text padding=\meta{dimension}}
    Sets all text paddings to \meta{dimension}.
\end{key}

\begin{key}{/tikz/data visualization/stack=\meta{dimension} (default 1em)}
    Shorthand for |tick text even padding=|\meta{dimension}.
    %
\begin{codeexample}[preamble={\usetikzlibrary{datavisualization.formats.functions}}]
\tikz \datavisualization [scientific axes,
                          all axes={length=2.5cm},
                          x axis={ticks={stack=1.5em}},
                          visualize as smooth line]
  data [format=function] {
    var y : interval[-100:100];
    func x = \value y*\value y;
  };
\end{codeexample}
    %
\end{key}

\begin{key}{/tikz/data visualization/stack'=\meta{dimension}}
    Shorthand for |tick text odd padding=|\meta{dimension}. The difference to
    |stack| is that the set of value that are ``lowered'' is exactly exchanged
    with the set of value ``lowered'' by |stack|.
    %
\begin{codeexample}[preamble={\usetikzlibrary{datavisualization.formats.functions}}]
\tikz \datavisualization [scientific axes,
                          all axes={length=2.5cm},
                          x axis={ticks=stack'},
                          visualize as smooth line]
  data [format=function] {
    var y : interval[-100:100];
    func x = \value y*\value y;
  };
\end{codeexample}
    %
\end{key}

Note that the above keys have an effect on all tick labels of an axis, also on
special ticks that you may have added using the |also at| key. When using the
|stack| key, you should specify a |tick text padding| explicitly for such keys:
%
\begin{codeexample}[
    width=7cm,
    preamble={\usetikzlibrary{datavisualization.formats.functions}},
]
\tikz \datavisualization
  [scientific axes,
   x axis={ticks={stack, many, major also at=
     {(pi) as [{tick text padding=2.5em}] $\pi$}}},
   visualize as smooth line]
  data [format=function] {
    var x : interval[0:(2*pi)];
    func y = sin(\value x r);
  };
\end{codeexample}


\subsubsection{Reference: Basic Strategies}
\label{section-dv-strategies}

\begin{key}{/tikz/data visualization/axis options/linear steps}
    This strategy places ticks at positions that are evenly spaced by the
    current value of |step|.

    In detail, the following happens: Let $a$ be the minimum value of the data
    values along the axis and let $b$ be the maximum. Let the current
    \emph{stepping} be $s$ (the stepping is set using the |step| option, see
    below) and let the current \emph{phasing} be $p$ (set using the |phase|)
    option. Then ticks are placed all positions $i\cdot s + p$ that lie in the
    interval $[a,b]$, where $i$ ranges over all integers.

    The tick positions computed in the way described above are \emph{major}
    step positions. In addition to these, if the key
    |minor steps between steps| is set to some number $n$, then $n$ many minor
    ticks are introduced between each two major ticks (and also before and
    after the last major tick, provided the values still lie in the interval
    $[a,b]$). Note that is $n$ is $1$, then one minor tick will be added in the
    middle between any two major ticks. Use a value of $9$ (not $10$) to
    partition the interval between two major ticks into ten equally sized minor
    intervals.
    %
\begin{codeexample}[preamble={\usetikzlibrary{datavisualization}}]
\begin{tikzpicture}
  \datavisualization
    [scientific axes={inner ticks, width=3cm},
     x axis={ticks={step=3, minor steps between steps=2}},
     y axis={ticks={step=.36}},
     visualize as scatter]
    data {
      x, y
      17, 30
      34, 32
    };
\end{tikzpicture}
\end{codeexample}
    %
\end{key}

\label{section-dv-exponential-strategy}

\begin{key}{/tikz/data visualization/axis options/exponential steps}
    This strategy produces ticks at positions that are appropriate for
    logarithmic plots. It is automatically selected when you use the
    |logarithmic| option with an axis.

    In detail, the following happens: As for |linear steps| let numbers $a$,
    $b$, $s$, and $p$ be given. Then, major ticks are placed at all positions
    $10^{i\cdot s+p}$ that lie in the interval $[a,b]$ for $i \in \mathbb{Z}$.

    The minor steps are added in the same way as for |linear steps|. In
    particular, they interpolate \emph{linearly} between major steps.
    %
\begin{codeexample}[preamble={\usetikzlibrary{datavisualization}}]
\begin{tikzpicture}
  \datavisualization
    [scientific axes,
     x axis={logarithmic, length=2cm, ticks={step=1.5}},
     y axis={logarithmic, ticks={step=1, minor steps between steps=9}},
     visualize as scatter]
    data {
      x, y
      1, 10
      1000, 1000000
    };
\end{tikzpicture}
\end{codeexample}
    %
\end{key}


\subsubsection{Advanced: Defining New Placement Strategies}

\begin{key}{/tikz/data visualization/axis options/tick placement strategy=\meta{macro}}
    This key can be used to install a so-called \emph{tick placement strategy}.
    Whenever |visualize ticks| is used to request some ticks to be visualized,
    it is checked whether some automatic ticks should be created. This is the
    case when the following key is set:
    %
    \begin{key}{/tikz/data visualization/compute step=\meta{code}}
        The \meta{code} should compute a suitable value for the stepping to be
        used by the \meta{macro} in the tick placement strategy.

        For instance, the |step| key sets |compute step| to
        |\def\tikz@lib@dv@step{#1}|. Thus, when you say |step=5|, then the
        desired stepping of |5| is communicated to the \meta{macro} via the
        macro |\tikz@lib@dv@step|.
    \end{key}

    Provided |compute step| is set to some nonempty value, upon visualization
    of ticks the \meta{macro} is executed. Typically, \meta{macro} will first
    call the \meta{code} stored in the key |compute step|. Then, it should
    implement some strategy then uses the value of the computed or desired
    stepping to create appropriate |at| commands. To be precise, it should set
    the keys |major|, |minor|, and/or |subminor| with some appropriate |at|
    values.

    Inside the call of \meta{macro}, the macro |\tikzdvaxis| will have been set
    to the name of the axis for which default ticks need to be computed. This
    allows you to access the minimum and the maximum value stored in the
    |scaling mapper| of that axis.
    %
\begin{codeexample}[width=7cm,preamble={\usetikzlibrary{datavisualization}}]
\def\silly{
  \tikzdatavisualizationset{major={at={
        2,3,5,7,11,13}}}
}
\begin{tikzpicture}
  \datavisualization [
    scientific axes, visualize as scatter,
    x axis={tick placement strategy=\silly}
    ]
    data {
      x, y
      0, 0
      15, 15
    };
\end{tikzpicture}
\end{codeexample}
    %
\end{key}


\subsection{Advanced: Creating New Axis Systems}

The |datavisualization| library comes with a number of predefined axis systems,
like |scientific axes=clean|, but it is also possible and to define new axis
systems. Doing so involves the following steps:
%
\begin{enumerate}
    \item Creating a number of axes.
    \item Configuring attributes of these axes like their length or default
        scaling.
    \item Creating visual representations of the axes.
    \item Creating visual representations of the ticks and grid lines.
\end{enumerate}

The first step uses |new ... axis| keys to create new axes, the last steps use
|visualize ...| keys to create the visual representations of the axes.

Note that the axis system has no control over the actual attribute value ranges
and neither over which ticks need to be drawn. The axis system can only provide
good defaults and then specify \emph{how} the ticks or labels should be drawn
and \emph{where} on the page -- but not at which values.

In the following, as a running example let us develop an axis system
|our system| that does the following: For the $x$-axis is looks like a normal
scientific axis system, but there are actually two $y$-axes: One at the left
and one at the right, each using a different attribute, but both coexisting in
the same picture.


\subsubsection{Creating the Axes}

A new axis system is created as a style key with the prefix
|/tikz/data visualization|. Thus, we would write:
%
\begin{codeexample}[code only]
\tikzset{
  data visualization/our system/.style={
    ...
  }
}
\end{codeexample}

In our system we need three axis: The $x$-axis, the left axis and the right
axis. Since all of these axes are Cartesian axes, we write the following:
%
\begin{codeexample}[code only]
\tikzset{
  data visualization/our system/.style={
    new Cartesian axis=x axis,
    new Cartesian axis=left axis,
    new Cartesian axis=right axis,
    x axis={attribute=x},
    left axis={unit vector={(0cm,1pt)}},
    right axis={unit vector={(0cm,1pt)}},
  }
}
\end{codeexample}
%
As can be seen, we also configure things so that the $x$-axis will use the |x|
attribute by default (users can later change this by saying
|x axis={attribute=|\meta{some other attribute}|}|), but we do not configure
the attributes of the |left axis| nor the |right axis|. We also make the left
and right axis point upward (the |x axis| needs no configuration here since a
Cartesian axis points right by default). The reason is the |left| would not be
a particularly good attribute name and this way we ensure that users have to
pick names themselves (hopefully good ones).

The next step is to define a standard scaling for the axes. Here, we can use
the same as for |scientific axes|, so we would add the following keys to the
definition of |our system|:
%
\begin{codeexample}[code only]
x axis    ={length=\pgfkeysvalueof{/tikz/data visualization/scientific axes/width}},
left axis ={length=\pgfkeysvalueof{/tikz/data visualization/scientific axes/height}},
right axis={length=\pgfkeysvalueof{/tikz/data visualization/scientific axes/height}}
\end{codeexample}

We now already have enough to try our system, although we will not yet see any
axes or ticks, but we will see the correct scaling of the attributes. Let us
first define a data group:
%
\begin{codeexample}[setup code]
\tikz \datavisualization data group {people and money} = {
    data [set=people 1] {
      time, people
      1900, 1000000000
      1920, 1500000000
      1930, 2000000000
      1980, 3000000000
    }
    data [set=people 2] {
      time, people
      1900, 2000000000
      1920, 2500000000
      1940, 4000000000
      2000, 5700000000
    }
    data [set=money 1] {
      time, money
      1910, 1.1
      1920, 2
      1930, 5
      1980, 2
    }
    data [set=money 2] {
      time, money
      1950, 3
      1960, 3
      1970, 4
      1990, 3.5
    }
  };
\end{codeexample}

\begin{codeexample}[setup code,hidden]
\tikzdatavisualizationset{
  our system/.style={
    new Cartesian axis=x axis,
    new Cartesian axis=left axis,
    new Cartesian axis=right axis,
    x axis={attribute=x},
    left axis={unit vector={(0cm,1pt)}},
    right axis={unit vector={(0cm,1pt)}},
    x axis    ={length=\pgfkeysvalueof{/tikz/data visualization/scientific axes/width}},
    left axis ={length=\pgfkeysvalueof{/tikz/data visualization/scientific axes/height}},
    right axis={length=\pgfkeysvalueof{/tikz/data visualization/scientific axes/height}}
  }
}
\end{codeexample}
%
\begin{codeexample}[preamble={\usetikzlibrary{datavisualization}}]
\tikz \datavisualization [
    our system,
    x axis={attribute=time, length=4cm},
    left axis ={attribute=money},
    right axis={attribute=people},
    visualize as line/.list={people 1, people 2, money 1, money 2},
    people 1={style={visualizer color=blue}},
    people 2={style={visualizer color=blue!50}},
    money 1={style={visualizer color=red}},
    money 2={style={visualizer color=red!50}}]
  data group {people and money};
\end{codeexample}


\subsubsection{Visualizing the Axes}
\label{section-dv-visualize-axis}

We must now show the axes themselves. For this we can use the |visualize axis|
key:

\begin{key}{/tikz/data visualization/axis options/visualize axis=\meta{options}}
    This key is passed to an axis as an option. It causes a visual
    representation of the axis to be created during the data visualization. The
    \meta{options} are used to determine where the axis should be drawn and how
    long it should be. We can specify, for instance, that an axis should be
    drawn at the minimum value of another axis or where another axis has the
    value |0|.


    \medskip
    \textbf{The goto, high, and low Keys.}
    In our example, the |left axis| should be shown at the left hand side. This
    is the position where the |x axis| has its minimum value. To specify this,
    we would use the following code:
    %
\begin{codeexample}[code only]
left axis={ visualize axis={ x axis={ goto=min } }
\end{codeexample}
    %
    As can be seen, we can pass another axis as an \meta{option} to
    |visualize axis|, where we pass the following key to the axis in turn:
    %
    \begin{key}{/tikz/data visualization/axis options/goto=\meta{value}}
        The key can be passed to an axis. It will set the attribute monitored
        by the axis to the given \meta{value}, which is usually some number.
        However, \meta{value} may also be one of the following, which causes a
        special behavior:
        %
        \begin{itemize}
            \item |min|: The attribute is set to the minimal value that the
                attribute has attained along this axis.
            \item |max|: Like |min|.
            \item |padded min|: This will also set the \meta{attribute}
                monitored by the axis to the same value as |min|.
                Additionally, however, the subkey
                |/data point/|\meta{attribute}|/offset| is set to the current
                padding for the minimum, see the description of |padding min|
                later on. The effect of this is that the actual point ``meant''
                by the attribute is offset by this padding along the
                attribute's axis.
            \item |padded max|: Like |padded min|.
        \end{itemize}
    \end{key}

    The |right axis| would be visualized the same way, only at |goto=max|. The
    $x$-axis actually needs to be visualized \emph{twice}: Once at the bottom
    and once at the top. Thus, we need to call |visualize axis| twice for this
    axis:
    %
\tikzdatavisualizationset{
  our system/.style={
    new Cartesian axis=x axis,
    new Cartesian axis=left axis,
    new Cartesian axis=right axis,
    x axis={attribute=x},
    left axis={unit vector={(0cm,1pt)}},
    right axis={unit vector={(0cm,1pt)}},
    x axis    ={length=\pgfkeysvalueof{/tikz/data visualization/scientific axes/width}},
    left axis ={length=\pgfkeysvalueof{/tikz/data visualization/scientific axes/height}},
    right axis={length=\pgfkeysvalueof{/tikz/data visualization/scientific axes/height}}
  }
}
\begin{codeexample}[
    preamble={\usetikzlibrary{datavisualization}},
    pre={\tikzdatavisualizationset{
  our system/.style={
    new Cartesian axis=x axis,
    new Cartesian axis=left axis,
    new Cartesian axis=right axis,
    x axis={attribute=x},
    left axis={unit vector={(0cm,1pt)}},
    right axis={unit vector={(0cm,1pt)}},
    x axis    ={length=\pgfkeysvalueof{/tikz/data visualization/scientific axes/width}},
    left axis ={length=\pgfkeysvalueof{/tikz/data visualization/scientific axes/height}},
    right axis={length=\pgfkeysvalueof{/tikz/data visualization/scientific axes/height}}
  }
}}]
\tikzset{
  data visualization/our system/.append style={
    left axis= {visualize axis={x axis=   {goto=min}}},
    right axis={visualize axis={x axis=   {goto=max}}},
    x axis=    {visualize axis={left axis={goto=min}},
                visualize axis={left axis={goto=max}}},
 }
}
\tikz \datavisualization [
    our system,
    x axis={attribute=time, length=4cm},
    left axis ={attribute=money},
    right axis={attribute=people},
    visualize as line/.list={people 1, people 2, money 1, money 2}]
  data group {people and money};
\end{codeexample}

    There is another key that is similar to |goto|, but has a slightly
    different semantics:
    %
    \begin{key}{/tikz/data visualization/axis options/goto pos=\meta{fraction}}
        The key works like |goto|, only the \meta{fraction} is not interpreted
        as a value but as a fraction of the way between the minimum and the
        maximum value for this axis.

        Suppose that for an axis the attribute range interval is $[500,1000]$
        and the reasonable interval is $[1,3]$. Then for a \meta{fraction} of
        |0|, the mapping process would choose value $1$ from the reasonable
        interval, for a \meta{fraction} of |1| the position $3$ from the
        reasonable interval, and for a \meta{fraction} or |0.25| the position
        $1.5$ since it is one quarter at the distance from $1$ to $3$.

        Note that neither the attribute range interval nor the transformation
        function for the attribute are important for the |goto pos| option --
        the \meta{fraction} is computed with respect to the reasonable
        interval. Also note that the values of the actual attribute
        corresponding to the fractional positions in the reasonable interval
        are not computed.
        %
\begin{codeexample}[
    preamble={\usetikzlibrary{datavisualization}},
    pre={\tikzdatavisualizationset{
  our system/.style={
    new Cartesian axis=x axis,
    new Cartesian axis=left axis,
    new Cartesian axis=right axis,
    x axis={attribute=x},
    left axis={unit vector={(0cm,1pt)}},
    right axis={unit vector={(0cm,1pt)}},
    x axis    ={length=\pgfkeysvalueof{/tikz/data visualization/scientific axes/width}},
    left axis ={length=\pgfkeysvalueof{/tikz/data visualization/scientific axes/height}},
    right axis={length=\pgfkeysvalueof{/tikz/data visualization/scientific axes/height}}
  }
}}]
\tikzset{
  data visualization/our system/.append style={
    x axis=    {visualize axis={left axis={goto pos=0.25}},
                visualize axis={left axis={goto pos=0.5}}},
 }
}
\tikz \datavisualization [
    our system,
    x axis={attribute=time, length=4cm},
    left axis ={attribute=money},
    right axis={attribute=people},
    visualize as line/.list={people 1, people 2, money 1, money 2}]
  data group {people and money};
\end{codeexample}
    \end{key}

    By default, when an axis is visualized, it spans the set of all possible
    values for the monitored attribute, that is, from |min| to |max|. However,
    there are actually two keys that allow you to adjust this:
    %
    \begin{key}{/tikz/data visualization/low=\meta{value}}
        This is the attribute value where the axis visualization starts. The
        same special values as for |goto| are permissible (like |min| or
        |padded min|, but also |0| or |1|).
    \end{key}
    %
    \begin{key}{/tikz/data visualization/high=\meta{value}}
        Like |low|, only for where the axis ends.
    \end{key}

    By default, |low=min| and |high=max| are set for an axis visualization.
    Another sensible setting is |low=padded min| and |high=padded max|. The
    following key provides a shorthand for this:
    %
    \begin{key}{/tikz/data visualization/padded}
        Shorthand for |low=padded min, high=padded max|.
    \end{key}
    %
    As an example, consider the |scientific axes=clean|. Here, each axis is
    actually drawn three times: Once at the minimum, once at the maximum and
    then once more at the padded minimum.


    \medskip
    \textbf{The axis line.}
    When an axis is drawn, \tikzname\ does not simply draw a straight line from
    the |low| position to the |high| position. In reality, the data
    visualization system uses the two commands |\pgfpathdvmoveto| and
    |\pgfpathdvlineto| internally. These will replace the straight line by a
    curve in certain situations. For instance, in a polar coordinate system, if
    an axis should be drawn along an angle axis for a fixed radius, an arc will
    be used instead of a straight line.


    \medskip
    \textbf{Styling the axis.}
    As can be seen, we now get the axis we want (but without the ticks,
    visualizing them will be explained later). The axis is, however, simply a
    black line. We can \emph{style} the axis in a manner similar to styling
    ticks and grid lines, see Section~\ref{section-dv-style}. In detail, the
    following styles get executed:
    %
    \begin{enumerate}
        \item |axis layer|
        \item |every axis|
        \item |styling|
    \end{enumerate}
    %
    Additionally, even before |every axis| is executed, |low=min| and
    |high=max| are executed.

    \begin{stylekey}{/tikz/data visualization/axis layer (initially on background layer)}
        The layer on which the axis is drawn. See the description of
        |grid layer| on page~\pageref{section-dv-grid-layer} for details.
    \end{stylekey}

    \begin{stylekey}{/tikz/data visualization/every axis}
        Put styling of the axis here. It is usually a good idea to set this
        style to |style={black!50}|.
    \end{stylekey}

    Recall that the |styling| key is set using the |style| key, see
    Section~\ref{section-dv-style}.
    %
% TODOsp: codeexamples: What is this empty `\tikzset` good for?
\tikzset{
}
\begin{codeexample}[
    preamble={\usetikzlibrary{datavisualization}},
    pre={\tikzdatavisualizationset{
  our system/.style={
    new Cartesian axis=x axis,
    new Cartesian axis=left axis,
    new Cartesian axis=right axis,
    x axis={attribute=x},
    left axis={unit vector={(0cm,1pt)}},
    right axis={unit vector={(0cm,1pt)}},
    x axis    ={length=\pgfkeysvalueof{/tikz/data visualization/scientific axes/width}},
    left axis ={length=\pgfkeysvalueof{/tikz/data visualization/scientific axes/height}},
    right axis={length=\pgfkeysvalueof{/tikz/data visualization/scientific axes/height}}
  }
}}]
\tikzset{
  data visualization/our system/.append style={
    every axis/.style={style=black!50}, % make this the default
    left axis= {visualize axis={x axis=   {goto=min}, style=red!75}},
    right axis={visualize axis={x axis=   {goto=max}, style=blue!75}},
    x axis=    {visualize axis={left axis={goto=min}},
                visualize axis={left axis={goto=max}}},
 }
}
\tikz \datavisualization [
    our system,
    x axis={attribute=time, length=4cm},
    left axis ={attribute=money},
    right axis={attribute=people},
    visualize as line/.list={people 1, people 2, money 1, money 2}]
  data group {people and money};
\end{codeexample}
\tikzset{
  data visualization/our system/.append style={
    every axis/.style={style=black!50}, % make this the default
    left axis= {visualize axis={x axis=   {goto=min}, style=red!75}},
    right axis={visualize axis={x axis=   {goto=max}, style=blue!75}},
    x axis=    {visualize axis={left axis={goto=min}},
                visualize axis={left axis={goto=max}}},
 }
}


    \medskip
    \textbf{Padding the Axis.}
    When an axis is visualized, it is often a good idea to make it ``a little
    bit longer'' or to ``remove it a bit from the border'', because the
    visualization of an axis should not interfere with the actual data. For
    this reason, a \emph{padding} can be specified for axes:

    \begin{key}{/tikz/data visualization/axis options/padding min=\meta{dimension}}
        This is the dimension that is used whenever |goto=padded min| is used.
        The \meta{dimension} is then put into the |offset| subkey of the
        attribute monitored by the axis. When a data point is transformed by a
        linear transformer and when this subkey is nonzero, this offset is
        added. (For an angle axis of a polar transformer, the \meta{dimension}
        is interpreted as an additional angle rather than as an additional
        distance). Note that \meta{dimension} should typically be negative
        since ``adding the \meta{dimension}'' will then make the axis longer
        (because it starts at a smaller value). The standard axis systems set
        the padding to some default and take its value into account:
        %
\begin{codeexample}[
    width=8cm,
    preamble={\usetikzlibrary{datavisualization.formats.functions}},
]
\begin{tikzpicture}
  \datavisualization [scientific axes=clean,
                      x axis={padding min=-1cm},
                      visualize as smooth line]
    data [format=function] {
      var x : interval [-3:5];
      func y = \value x * \value x;
    };
\end{tikzpicture}
\end{codeexample}

        Using padded and using the |padded| key, we can visualize our axis ``a
        little removed from the actual data'':
        %
\begin{codeexample}[
    preamble={\usetikzlibrary{datavisualization}},
    pre={\tikzdatavisualizationset{
  our system/.style={
    new Cartesian axis=x axis,
    new Cartesian axis=left axis,
    new Cartesian axis=right axis,
    x axis={attribute=x},
    left axis={unit vector={(0cm,1pt)}},
    right axis={unit vector={(0cm,1pt)}},
    x axis    ={length=\pgfkeysvalueof{/tikz/data visualization/scientific axes/width}},
    left axis ={length=\pgfkeysvalueof{/tikz/data visualization/scientific axes/height}},
    right axis={length=\pgfkeysvalueof{/tikz/data visualization/scientific axes/height}}
  }
}%
\tikzset{
  data visualization/our system/.append style={
    every axis/.style={style=black!50}, % make this the default
    left axis= {visualize axis={x axis=   {goto=min}, style=red!75}},
    right axis={visualize axis={x axis=   {goto=max}, style=blue!75}},
    x axis=    {visualize axis={left axis={goto=min}},
                visualize axis={left axis={goto=max}}},
 }
}}]
\tikzset{
  data visualization/our system/.append style={
    all axes=  {padding=.5em},
    left axis= {visualize axis={x axis=   {goto=padded min}, padded}},
    right axis={visualize axis={x axis=   {goto=padded max}, padded}},
    x axis=    {visualize axis={left axis={goto=padded min}, padded},
                visualize axis={left axis={goto=padded max}, padded}},
 }
}
\tikz \datavisualization [
    our system,
    x axis={attribute=time, length=3cm},
    left axis ={attribute=money},
    right axis={attribute=people},
    visualize as line/.list={people 1, people 2, money 1, money 2}]
  data group {people and money};
\end{codeexample}
    \end{key}

    \begin{key}{/tikz/data visualization/axis options/padding max=\meta{dimension}}
        Works like |padding min|, but \meta{dimension} should typically be
        positive.
    \end{key}

    \begin{key}{/tikz/data visualization/axis options/padding=\meta{dimension}}
        Sets both |padding min| to the negated value of \meta{dimension} and
        |padding max| to \meta{dimension}.
    \end{key}
\end{key}


\subsubsection{Visualizing Grid Lines}
\label{section-dv-visualize-gridlines}

As explained earlier, the |grid| key is used to specify at which positions grid
lines should be drawn in principle. However, this key does not actually cause
any grid lines to be drawn. Instead, the |visualize grid| key is used by the
axis system to specify how grid lines are drawn.

\begin{key}{/tikz/data visualization/axis options/visualize grid=\meta{options}}
    This key is passed to an axis. It causes grid lines to be drawn at the
    positions specified by the |grid| key for this axis. The \meta{options}
    govern where and how the grid lines will be drawn.


    \medskip
    \textbf{The direction axis.}
    At first sight, one might expect that the grid lines for an axis should
    simply be drawn perpendicular to the axis between the minimum and maximum
    value of the axis. However, things are somewhat more difficult in reality:
    %
    \begin{enumerate}
        \item A grid line is supposed to indicate all positions where a certain
            attribute attains a fixed value. But, then, a grid line does not
            really need to be a grid \emph{line}. Consider for instance a three
            dimensional axis system. A ``grid line'' for the $x$-coordinate |3|
            would actually be a ``grid plane''.
        \item For a polar coordinate  system and a fixed radius, this set of
            positions at a certain radius is not a straight line, but an arc.
            For more complicated coordinate systems such as the one arising
            from three-dimensional spherical projections, a grid line may well
            be a fairly involved curve.
    \end{enumerate}
    %
    The |visualize grid| command addresses these complications as follows:
    %
    \begin{enumerate}
        \item A grid line is always a line, not a plane or a volume. This means
            that in the example of a three dimensional axis system and the
            $x$-attribute being |3|, one would have to choose whether the grid
            line should go ``along'' the $y$-axis or ``along'' the $z$-axis for
            this position. One can, however, call the |visualize grid| command
            twice, once for each direction, to cause grid lines to be shown for
            both directions.
        \item A grid line is created by moving to a start position and then
            doing a lineto to the target position. However, the ``moveto'' and
            ``lineto'' are done by calling special commands of the data
            visualization system. These special commands allow coordinate
            system to ``notice'' that the line is along an axis and will allow
            them to replace the straight line by an appropriate curve. The
            polar axes systems employ this strategy, for instance.
    \end{enumerate}

    By the above discussion, in order to create a grid line for attribute $a$
    having value $v$, we need to specify an axis ``along'' which the line
    should be drawn. When there  are only two axes, this is usually ``the other
    axis''. This ``other axis'' is specified using the following key:
    %
    \begin{key}{/tikz/data visualization/direction axis=\meta{axis name}}
        You must pass this key as an \meta{option} each time you use
        |visualize axis|. When the grid line is drawn, the attribute $a$ is set
        to $v$ and the axis \meta{axis name}'s attribute is set once to the
        current value of |low| and once to |high|. Then a line is drawn between
        these two positions using |\pgfpathdvlineto|.
    \end{key}
    %
    The |low| and |high| keys are the same as the ones used in the
    |visualize axis| key.
    %
\begin{codeexample}[preamble={\usetikzlibrary{datavisualization}}]
\tikz \datavisualization [
    xyz Cartesian cabinet,
    all axes={visualize axis={low=0, style=->}},
    x axis={visualize grid={direction axis=y axis}, grid=many},
    visualize as scatter]
  data {
    x, y, z
    0, 0, 1
    0, 1, 0
    2, 2, 2
  };
\end{codeexample}
    %
\begin{codeexample}[preamble={\usetikzlibrary{datavisualization}}]
\tikz \datavisualization [
    xyz Cartesian cabinet,
    all axes={visualize axis={low=0, style=->}, grid=many},
    x axis={visualize grid={direction axis=z axis}},
    z axis={visualize grid={direction axis=x axis},
            visualize grid={direction axis=y axis},},
    visualize as scatter]
  data {
    x, y, z
    0, 0, 1
    0, 1, 0
    2, 2, 2
  };
\end{codeexample}


    \medskip
    \textbf{Styling the grid lines.}
    When a grid line is draw, styles are applied as described in
    Section~\ref{section-dv-styling-grid-lines}.


    \medskip
    \textbf{The major, minor, and subminor grid lines.}
    The |grid| option allows you to specify for each kind of grid line (major,
    minor, or subminor) a set of different values for which these grid lines
    should be drawn. Correspondingly, it is also possible to configure for each
    kind of grid line how it should be drawn. For this, the |major|, |minor|,
    |subminor|, and also the |common| keys can be used inside the
    \meta{options} of |visualize grid|. While as option to |grid| these keys
    are used to specify |at| values, as options of |visualize grid| they are
    used to configure the different kinds of grid lines.

    Most of the time, no special configuration is necessary since all styling
    is best done by configuring keys like |every major grid|. You need to use a
    key like |major| only if you wish to configure for instance the |low| or
    |high| values of a |major| grid line differently from those of |minor| grid
    lines -- are rather unlikely setting -- or when the styling should deviate
    from the usual settings.
    %
\begin{codeexample}[preamble={\usetikzlibrary{datavisualization}}]
\tikz \datavisualization [
    xy Cartesian,
    all axes={visualize axis={low=0, style=->},
              grid={some, minor steps between steps}},
    x axis=  {visualize grid={
                direction axis=y axis,
                minor={low=0.25, high=1.75, style=red!50}}},
    visualize as scatter]
  data {
    x, y
    0, 0
    3, 3
  };
\end{codeexample}
    %
\end{key}

Returning to the example of |our system| with the two axis systems, it is
straight-forward to configure the grid lines of the $x$-axis: The direction
axis is either of the other two axis (they point in the same direction and they
have the same range). For the other two axes, we visualize one grid
independently of the other, using different colors.
%
\begin{codeexample}[preamble={\usetikzlibrary{datavisualization}}]
\tikzset{
  data visualization/our system/.append style={
    x axis=    {visualize grid={direction axis=left axis}},
    left axis= {visualize grid={direction axis=x axis,
                                common={style=red!50}}},
    right axis={visualize grid={direction axis=x axis,
                                common={style=blue!50}}},
  }
}
\tikz \datavisualization [
    our system,
    x axis={attribute=time, length=3cm, grid=many},
    left axis ={attribute=money, grid=some},
    right axis={attribute=people, grid=few},
    visualize as line/.list={people 1, people 2, money 1, money 2}]
  data group {people and money};
\end{codeexample}


\subsubsection{Visualizing the Ticks and Tick Labels}
\label{section-dv-visualize-ticks}

\begin{key}{/tikz/data visualization/axis options/visualize ticks=\meta{options}}
    Visualizing a tick involves (possibly) drawing a tick mark and adding
    (possibly) the tick node. The process is similar to |visualize grid|: Users
    use the |ticks| key to configure how many ticks they would like for an axis
    and at which positions. The axis system uses the |visualize ticks| key to
    specify where these ticks should actually be shown.

    Unlike grid lines, which are typically only visualized once for each
    combination of an axis and a direction axis, tick marks might be visualized
    at different places for the same axis. Consider for instance the
    |scientific axes|:
    %
\begin{codeexample}[preamble={\usetikzlibrary{datavisualization.formats.functions}}]
\tikz \datavisualization [scientific axes, all axes={length=3cm},
                          x axis={ticks={stack}},
                          visualize as smooth line]
  data [format=function] {
    var x : interval [0:2];
    func y = \value  x*\value x;
  };
\end{codeexample}
    %
    Have a look at the ticks on the $y$-axis: There are ticks at values |0|,
    |1|, |2|, |3|, and~|4|. These are visualized both at the left side (where
    the tick nodes are also shown) and additionally also at the right side, but
    only as small marks. Similarly, the ticks on the $x$-axis appear at the
    bottom, but also (in much simpler versions) at the top. Both for the
    $x$-axis and for the $y$-axis the |visualize ticks| key was called twice.


    \medskip
    \textbf{The tick marks.}
    Drawing a tick mark is quite similar to visualizing a grid line; indeed a
    tick mark can be thought of as a ``mini grid line'': Just like a grid line
    it ``points a long an axis''. However, a tick will always be a short
    straight line -- even when the coordinate system is actually twisted
    (experimentation has shown that ticks that follow the curvature of the
    coordinate system like grid lines are hard to recognize). For this reason,
    the |low| and |high| keys have a different meaning from the one used with
    the |visualize grid| key. In detail to configure the size and position of a
    tick mark for the value $v$ of attribute $a$, proceed as follows:
    %
    \begin{itemize}
        \item The |visualize ticks| key will have setup attribute $a$ to be
            equal to $v$.
        \item You should now use the |goto| or |goto pos| key together with all
            \emph{other} axes to configure at which position with respect to
            these other options the tick mark should be shown. For instance,
            suppose we want tick marks in |our system| for the $x$-axis at the
            bottom and at the top. This corresponds to once setting the
            |left axis| to its minimal value and once to its maximal value:
            %
\begin{codeexample}[preamble={\usetikzlibrary{datavisualization}}]
\tikzset{
  data visualization/our system/.append style={
    x axis={visualize ticks={direction axis=left axis, left axis={goto=min}},
            visualize ticks={direction axis=left axis, left axis={goto=max}},
    }
  }
}
\tikz \datavisualization [
    our system,
    x axis={attribute=time, length=3cm, ticks=many},
    left axis ={attribute=money},
    right axis={attribute=people},
    visualize as line/.list={people 1, people 2, money 1, money 2}]
  data group {people and money};
\end{codeexample}
            %
        \item In the above example, we may wish to shorten the ticks a bit at
            the bottom and at the top. For this, we use the |low| and |high|
            key:
            %
            \begin{key}{/tikz/data visualization/low=\meta{dimension}}
                When used with the |visualize ticks| option, the |low| key
                contains a dimension that specifies the extend of the tick
                going ``toward the minimum'' of the direction axis. More
                precisely, when a tick mark is visualized, a unit tangent
                vector at the current data point in the direction of the
                |direction axis| is computed and this vector is multiplied by
                \meta{dimension} to compute the start position of the tick
                line. The end position is given by this vector times the |high|
                value.

                Note that the \meta{dimension} should usually be negative for
                the |low| key and positive for the |high| key.

                For tick marks where a tick label node is shown, the
                \meta{dimension} is increased by the current values of keys
                like |tick text even low padding|, see
                Section~\ref{section-dv-stacking} for details.
            \end{key}
            %
            \begin{key}{/tikz/data visualization/high=\meta{dimension}}
                Like |low|.
            \end{key}
            %
            \begin{key}{/tikz/data visualization/tick length=\meta{dimension}}
                Shorthand for |low=-|\meta{dimension}|, high=|\meta{dimension}.
            \end{key}

            What we want to happen is that in the upper visualization of the
            ticks the |low| value is |0pt|, while in the lower one the |high|
            value is |0pt|:
            %
\begin{codeexample}[preamble={\usetikzlibrary{datavisualization}}]
\tikzset{
  data visualization/our system/.append style={
    x axis={
      visualize ticks={direction axis=left axis,high=0pt,left axis={goto=min}},
      visualize ticks={direction axis=left axis,low=0pt,left axis={goto=max}},
    }
  }
}
\tikz \datavisualization [
    our system,
    x axis={attribute=time, length=3cm, ticks=many},
    left axis ={attribute=money},
    right axis={attribute=people},
    visualize as line/.list={people 1, people 2, money 1, money 2}]
  data group {people and money};
\end{codeexample}
            %
    \end{itemize}
    %
    In order to style the tick mark, use the styling mechanism that is detailed
    in Section~\ref{section-dv-styling-ticks}.


    \medskip
    \textbf{The tick label node.}
    At certain tick positions, we may wish to add a node indicating the value
    of the attribute at the given position. The |visualize ticks| command has
    no influence over which text should be shown at a node -- the text is
    specified and typeset as explained in Section~\ref{section-dv-tick-labels}.

    Each time |visualize ticks|, for each tick position up to two tick label
    nodes will be created: One at the |low| position and one at the |high|
    position. The following keys are used to configure which of these cases
    happen:
    %
    \begin{key}{/tikz/data visualization/tick text at low=\opt{\meta{true or false}} (default true)}
        Pass this option to |visualize ticks| when you want tick label nodes to
        be placed at the |low| position of each tick mark.
        %
\begin{codeexample}[preamble={\usetikzlibrary{datavisualization}}]
\tikzset{
  data visualization/our system/.append style={
    x axis={
      visualize ticks={direction axis=left axis, left axis={goto=min},
                       high=0pt, tick text at low, stack},
      visualize ticks={direction axis=left axis, left axis={goto=max},
                       low=0pt, tick text at high, stack}
    }
  }
}
\tikz \datavisualization [
    our system,
    x axis={attribute=time, length=3cm, ticks=some},
    left axis ={attribute=money},
    right axis={attribute=people},
    visualize as line/.list={people 1, people 2, money 1, money 2}]
  data group {people and money};
\end{codeexample}
    \end{key}
    %
    \begin{key}{/tikz/data visualization/tick text at high=\opt{\meta{true or false}} (default true)}
        Like |tick text at low|.
    \end{key}

    \begin{key}{/tikz/data visualization/no tick text}
        Shorthand for |tick text at low=false, tick text at high=false|.
        %
\begin{codeexample}[preamble={\usetikzlibrary{datavisualization.formats.functions}}]
\tikz \datavisualization [scientific axes, all axes={length=3cm},
                          x axis={ticks={
                              major also at={6.5 as [no tick text]}}},
                          visualize as smooth line]
  data [format=function] {
    var x : interval [5:10];
    func y = \value x * \value x;
  };
\end{codeexample}
    \end{key}

    When a tick label node is to be placed at the low or the high position, the
    next step is to determine the exact position and the correct anchor of the
    node. This is done as follows:
    %
    \begin{itemize}
        \item In order to compute an appropriate |anchor|, the tick mark is
            considered: This is a short line pointing in a certain direction.
            For a tick label node at the |low| position, the |anchor| attribute
            is setup in such a way that the node label will be below the |low|
            position when the tick mark direction points up, it will be to the
            right when the direction points left, above when it points down,
            and so on also for diagonal directions. Similarly, for the |high|
            position, when the direction points up, the node will be placed
            above the tick mark and so on.

            This computation is done automatically.
        \item The tick label node is styled. The styles that are applied are
            described in Section~\ref{section-dv-styling-ticks}.
        \item A tick label node for the |low| position is usually anchored at
            this |low| position, but an additional padding will be added as
            described in Section~\ref{section-dv-stacking}.
    \end{itemize}
\end{key}


\subsubsection{Visualizing the Axis Labels}
\label{section-dv-visualize-label}

The |label| option can be used with an axis to specify a text should be shown
next to the axis to indicates which attribute this axis refers to. Like |ticks|
or |grid|, the |label| option does not actually draw the label, this is the job
of the |visualize label| key, which is configured by the axis system.

\begin{key}{/tikz/data visualization/axis options/visualize label=\meta{options}}
    The \meta{options} should be used to configure a ``good place'' for the
    axis label. Usually, you will use the |goto| or the |goto pos| key.

    For the example of |our system|, we would like the label of the |x axis| to
    be placed below at the middle of the axis, so we use |goto pos=.5| to
    determine this position. Concerning the other axes, we want it to be placed
    at the minimum position of the |left axis| with a lot of padding.
    %
\begin{codeexample}[width=7cm,preamble={\usetikzlibrary{datavisualization}}]
\tikzdatavisualizationset{
  our system/.append style={
    x axis={visualize label={
        x axis={goto pos=.5},
        left axis={padding=1.5em, goto=padded min}}}
  }
}
\tikz \datavisualization [
    our system,
    x axis={attribute=time, ticks=some, label},
    left axis ={attribute=money},
    right axis={attribute=people},
    visualize as line/.list={
      people 1, people 2, money 1, money 2}]
  data group {people and money};
\end{codeexample}

    In the above example, the |padding| of |1.5em| was rather arbitrary and
    ``suboptimal''. It would be outright wrong if the labels on the |x axis|
    were larger or if they were missing. It would be better if the vertical
    position of the |x axis| label were always ``below'' all other options. For
    such cases a slightly strange approach is useful: You position the node
    using |node style={at=...}| where |at| is now the normal \tikzname\ option
    that is used to specify the position of a node. Inside the |...|, you
    specify that the horizontal position should be the bottom of
    up-to-now-constructed data visualization and the vertical position should
    be at the ``origin'', which is, however, the position computed by the
    |goto| keys for the axes:
    %
\begin{codeexample}[width=7cm,preamble={\usetikzlibrary{datavisualization}}]
\tikzdatavisualizationset{
  our system/.append style={
    x axis={visualize label={
      x axis={goto pos=.5},
      node style={
        at={(0,0 |- data visualization bounding box.south)},
        below
} } } } }
\tikz \datavisualization [
    our system,
    x axis={attribute=time, ticks=some, label=Year},
    left axis ={attribute=money},
    right axis={attribute=people},
    visualize as line/.list={
      people 1, people 2, money 1, money 2}]
  data group {people and money};
\end{codeexample}

    Two additional keys are useful for positioning axis labels:
    %
    \begin{key}{/tikz/data visualization/axis option/anchor at min}
        When passed to an axis, this key sets the |anchor| so that a node
        positioned at either the |min| or the |padded min| value of the axis
        will be placed ``nicely'' with respect to the axis. For instance, if
        the axis points upwards from the |min| value to the |max| value, the
        |anchor| would be set to |north| since this gives a label below the
        axis's start. Similarly, if the axis points right, the anchor would be
        set to |east|, and so on.
    \end{key}
    %
    \begin{key}{/tikz/data visualization/axis option/anchor at max}
        Like |anchor at min|.
    \end{key}
\end{key}


\subsubsection{The Complete Axis System}

Here is the code for the complete axis system developed above and an example of
how it is used:
%
\begin{codeexample}[code only]
\tikzdatavisualizationset{ our system/.style={
  % The axes
  new Cartesian axis=x axis,     new Cartesian axis=left axis,         new Cartesian axis=right axis,
  % The directions of the axes
  all axes={padding=.5em},       left axis={unit vector={(0cm,1pt)}},  right axis={unit vector={(0cm,1pt)}},
  % The default attributes, other attributes must be configured
  x axis={attribute=x},
  % The lengths of the axes
  x axis    ={length=\pgfkeysvalueof{/tikz/data visualization/scientific axes/width}},
  left axis ={length=\pgfkeysvalueof{/tikz/data visualization/scientific axes/height}},
  right axis={length=\pgfkeysvalueof{/tikz/data visualization/scientific axes/height}},
  % The styling of the axes
  every axis/.style={style=black!50}, % make this the default
  % Visualizing the axes themselves
  left axis= {visualize axis={x axis=   {goto=padded min}, style=red!75, padded}},
  right axis={visualize axis={x axis=   {goto=padded max}, style=blue!75,padded}},
  x axis=    {visualize axis={left axis={goto=padded min}, padded},
              visualize axis={left axis={goto=padded max}, padded}},
  % Visualizing the grid, when requested
  x axis=    {visualize grid={direction axis=left axis}},
  left axis= {visualize grid={direction axis=x axis, common={style=red!50}}},
  right axis={visualize grid={direction axis=x axis, common={style=blue!50}}},
  % Visualizing the ticks, when requested
  left axis={visualize ticks={style={red!50!black}, direction axis=x axis,
                              x axis={goto=padded min}, high=0pt, tick text at low}},
  right axis={visualize ticks={style={blue!80!black}, direction axis=x axis,
                              x axis={goto=padded max}, low=0pt, tick text at high}},
  x axis={visualize ticks={direction axis=left axis, left axis={goto=padded min}, high=0pt,
                           tick text at low},
          visualize ticks={direction axis=left axis, left axis={goto=padded max}, low=0pt}},
  % By default, there are ticks on all axes
  all axes={ticks},
  % Visualizing the axis labels, when requested
  x axis={visualize label={x axis={goto pos=.5}, node style={
        at={(0,0 |- data visualization bounding box.south)}, below}}},
  left axis={visualize label={left axis={goto pos=.5}, node style={
        at={(0,0 -| data visualization bounding box.west)}, rotate=90, anchor=south, red!50!black}}},
  right axis={visualize label={right axis={goto pos=.5}, node style={
        at={(0,0 -| data visualization bounding box.east)}, rotate=-90, anchor=south,  blue!80!black}}},
}}
\end{codeexample}

\begin{codeexample}[
    preamble={\usetikzlibrary{datavisualization}},
    pre={\tikzdatavisualizationset{
  our system/.style={
    % The axes
    new Cartesian axis=x axis,
    new Cartesian axis=left axis,
    new Cartesian axis=right axis,
    % The default attributes, other attributes must be configured
    x axis={attribute=x},
    % The directions of the axes
    all axes={padding=.5em},
    left axis={unit vector={(0cm,1pt)}},
    right axis={unit vector={(0cm,1pt)}},
    % The lengths of the axes
    x axis    ={length=\pgfkeysvalueof{/tikz/data visualization/scientific axes/width}},
    left axis ={length=\pgfkeysvalueof{/tikz/data visualization/scientific axes/height}},
    right axis={length=\pgfkeysvalueof{/tikz/data visualization/scientific axes/height}},
    % The styling of the axes
    every axis/.style={style=black!50}, % make this the default
    % Visualizing the axes themselves
    left axis= {visualize axis={x axis=   {goto=padded min}, style=red!75, padded}},
    right axis={visualize axis={x axis=   {goto=padded max}, style=blue!75,padded}},
    x axis=    {visualize axis={left axis={goto=padded min}, padded},
                visualize axis={left axis={goto=padded max}, padded}},
    % Visualizing the grid, when requested
    x axis=    {visualize grid={direction axis=left axis, padded}},
    left axis= {visualize grid={direction axis=x axis, padded, common={style=red!50}}},
    right axis={visualize grid={direction axis=x axis, padded, common={style=blue!50}}},
    % Visualizing the ticks, when requested
    left axis={
      visualize ticks={style={red!50!black}, direction axis=x axis, x axis={goto=padded min}, high=0pt, tick text at low}},
    right axis={
      visualize ticks={style={blue!80!black}, direction axis=x axis, x axis={goto=padded max}, low=0pt, tick text at high}},
    x axis={
      visualize ticks={direction axis=left axis, left axis={goto=padded min}, high=0pt, tick text at low},
      visualize ticks={direction axis=left axis, left axis={goto=padded max}, low=0pt}
    },
    % By default, there are ticks on all axes
    all axes={ticks},
    % Visualizing the axis labels, when requested
    x axis={visualize label={
      x axis={goto pos=.5}, node style={at={(0,0 |- data visualization bounding box.south)}, below}}},
    left axis={visualize label={
      left axis={goto pos=.5}, node style={
        at={(0,0 -| data visualization bounding box.west)}, rotate=90, anchor=south, red!50!black}}},
    right axis={visualize label={
      right axis={goto pos=.5}, node style={
        at={(0,0 -| data visualization bounding box.east)}, rotate=-90, anchor=south,  blue!80!black}}},
  }
}}]
\tikz \datavisualization [
  our system,
  x axis={attribute=time, label=Year,
    ticks={tick text padding=2pt,  style={/pgf/number format/set thousands separator=}}},
  left axis={attribute=money, label=Spending,
    padding min=0, include value=0, grid,
    ticks={tick prefix=\$, style={/pgf/number format/fixed,
        /pgf/number format/fixed zerofill, /pgf/number format/precision=2}}},
  right axis={attribute=people,
    label=Population,
    padding min=0, include value=0,
    ticks={style=/pgf/number format/fixed}},
  visualize as line/.list={
    people 1, people 2, money 1, money 2},
  people 1={style={visualizer color=blue}},
  people 2={style={visualizer color=blue!50}},
  money 1={style={visualizer color=red}},
  money 2={style={visualizer color=red!50}} ]
data group {people and money};
\end{codeexample}


\subsubsection{Using the New Axis System Key}

The axis system |our system| that we developed in the course of the previous
section is not yet ``configurable''. The only configuration that was possible
was to ``misuse'' the |width| and |height| keys of the |scientific axes|.

In order to make |our system| configurable so that we can say
|our system=|\meta{options}, where \meta{options} are executed with the path
prefix
%
\begin{codeexample}[code only]
/tikz/data visualization/our system
\end{codeexample}
%
we can use the following key:

\begin{key}{/tikz/data visualization/new axis system=\marg{axis system
      name}\marg{axis setup}\marg{default options}\\ \marg{application
      options}%
}
    The |new axis system| key takes four parameters. The first one,
    \meta{system name}, is the name of the to-be-created axis system,
    |our system| in our case. The |new axis system| will create the following
    new key:
    %
    \begin{key}{/tikz/data visualization/\meta{axis system name}=\opt{\meta{options}}}
        When the key \meta{axis system name} is used, the following keys will be
        executed in the following order:
        %
        \begin{enumerate}
            \item The \meta{axis setup} with the path prefix
                |/tikz/data visualization/|.
            \item The \meta{default options} with the same path prefix.
            \item The following style:
                %
                \begin{stylekey}{/tikz/data visualization/every \meta{axis system name}}
                    Even though this style has the path prefix
                    |/tikz/data visualization| itself, the keys stored in this
                    style will be executed with the path prefix
                    |/tikz/data visualization/|\meta{axis system name}.
                \end{stylekey}
            \item The \meta{options} with the path prefix
                |/tikz/data visualization/|\meta{axis system name}.
            \item The \meta{application options} with the path prefix
                |/tikz/data visualization/|
        \end{enumerate}
    \end{key}

    Let us now have a look at what all of this means. First, the \meta{axis
    setup} will contain all options that setup the axis system in all ways that
    need not be configured. For instance, the \meta{axis setup} for the
    |scientific axes| will create an |x axis| and also a |y axis| (because
    these are always present), but will not setup the label visualization
    (because this can be configured in different ways). For |our system|, which
    cannot be configured at all, we would place all of our configuration in the
    \meta{axis setup}.

    The \meta{default options} can be used to pick default values that would
    usually be passed to the \meta{options} of the newly created axis system.
    For instance, for |scientific axis|, the \meta{default options} are set to
    |outer ticks,standard labels|, because these are the defaults.

    Finally, the \meta{application options} can be used to actually apply the
    configuration that has been chosen by the \meta{options}. The idea is that
    \meta{default options}, \meta{options}, and also |every| \meta{axis system
    name} all have a chance of changing, re-changing and re-setting all sorts
    of styles and keys. Then, with the last change ``winning'', the resulting
    setting of a style can be executed, which may then cause a label
    visualization to be installed.
\end{key}
